\documentclass[a4paper,
fontsize=11pt,
%headings=small,
oneside,
numbers=noperiodatend,
parskip=half-,
bibliography=totoc,
final
]{scrartcl}

\usepackage[babel]{csquotes}
\usepackage{synttree}
\usepackage{graphicx}
\setkeys{Gin}{width=.4\textwidth} %default pics size

\graphicspath{{./plots/}}
\usepackage[ngerman]{babel}
\usepackage[T1]{fontenc}
%\usepackage{amsmath}
\usepackage[utf8x]{inputenc}
\usepackage [hyphens]{url}
\usepackage{booktabs} 
\usepackage[left=2.4cm,right=2.4cm,top=2.3cm,bottom=2cm,includeheadfoot]{geometry}
\usepackage{eurosym}
\usepackage{multirow}
\usepackage[ngerman]{varioref}
\setcapindent{1em}
\renewcommand{\labelitemi}{--}
\usepackage{paralist}
\usepackage{pdfpages}
\usepackage{lscape}
\usepackage{float}
\usepackage{acronym}
\usepackage{eurosym}
\usepackage{longtable,lscape}
\usepackage{mathpazo}
\usepackage[normalem]{ulem} %emphasize weiterhin kursiv
\usepackage[flushmargin,ragged]{footmisc} % left align footnote
\usepackage{ccicons} 
\setcapindent{0pt} % no indentation in captions

%%%% fancy LIBREAS URL color 
\usepackage{xcolor}
\definecolor{libreas}{RGB}{112,0,0}

\usepackage{listings}

\urlstyle{same}  % don't use monospace font for urls

\usepackage[fleqn]{amsmath}

%adjust fontsize for part

\usepackage{sectsty}
\partfont{\large}

%Das BibTeX-Zeichen mit \BibTeX setzen:
\def\symbol#1{\char #1\relax}
\def\bsl{{\tt\symbol{'134}}}
\def\BibTeX{{\rm B\kern-.05em{\sc i\kern-.025em b}\kern-.08em
    T\kern-.1667em\lower.7ex\hbox{E}\kern-.125emX}}

\usepackage{fancyhdr}
\fancyhf{}
\pagestyle{fancyplain}
\fancyhead[R]{\thepage}

% make sure bookmarks are created eventough sections are not numbered!
% uncommend if sections are numbered (bookmarks created by default)
\makeatletter
\renewcommand\@seccntformat[1]{}
\makeatother

% typo setup
\clubpenalty = 10000
\widowpenalty = 10000
\displaywidowpenalty = 10000

\usepackage{hyperxmp}
\usepackage[colorlinks, linkcolor=black,citecolor=black, urlcolor=libreas,
breaklinks= true,bookmarks=false,bookmarksopen=false]{hyperref}
\usepackage{breakurl}

%meta
%meta

\fancyhead[L]{C. Krause \\ %author
LIBREAS. Library Ideas, 36 (2019). % journal, issue, volume.
\href{http://nbn-resolving.de/}
{}} % urn 
% recommended use
%\href{http://nbn-resolving.de/}{\color{black}{urn:nbn:de...}}
\fancyhead[R]{\thepage} %page number
\fancyfoot[L] {\ccLogo \ccAttribution\ \href{https://creativecommons.org/licenses/by/4.0/}{\color{black}Creative Commons BY 4.0}}  %licence
\fancyfoot[R] {ISSN: 1860-7950}

\title{\LARGE{Was ist Informationswissenschaft und wenn ja, wie viele?}} % title
\subtitle{\Large{Ein Veranstaltungsbericht}}
\author{Carmen Krause} % author

\setcounter{page}{1}

\hypersetup{%
      pdftitle={Was ist Informationswissenschaft und wenn ja, wie viele? Ein Veranstaltungsbericht},
      pdfauthor={Carmen Krause},
      pdfcopyright={CC BY 4.0 International},
      pdfsubject={LIBREAS. Library Ideas, 36 (2019).},
      pdfkeywords={Informationswissenschaft, Zukunft, BAK, Veranstaltungsbericht},
      pdflicenseurl={https://creativecommons.org/licenses/by/4.0/},
      pdfcontacturl={http://libreas.eu},
      baseurl={http://libreas.eu},
      pdflang={de},
      pdfmetalang={de}
     }



\date{}
\begin{document}

\maketitle
\thispagestyle{fancyplain} 

%abstracts

%body
Gegenstand und Zukunftsfähigkeit der Informationswissenschaft geben
bereits seit Jahrzehnten Anlass zur Diskussion. Als Fortsetzung dieser
Diskussion erschien im April 2019 beim Simon Verlag für
Bibliothekswissen das Sammelwerk \enquote{Zukunft der
Informationswissenschaft : Hat die Informationswissenschaft eine
Zukunft?}.\footnote{\url{http://www.simon-bw.de/books/bibliothekswissenschaft/item/zukunft-der-informationswissenschaft-hat-die-informationswissenschaft-eine-zukunft}
  {[}zuletzt gesehen am 14.10.2019{]}} Das Sammelwerk ist eine
themenbezogene Zusammenstellung von Einzelbeiträgen, die fortlaufend im
Newsletter von Open Password\footnote{\url{http://www.password-online.de/push-dienst-archiv}
  {[}zuletzt gesehen am 14.10.2019{]}} publiziert wurden beziehungsweise
werden.

Am 05. September 2019 lud der Herausgeber von Sammelwerk und Newsletter,
Willi Bredemeier, in Kooperation mit dem Berliner Arbeitskreis (BAK)
Information in die Universitätsbibliothek der Technischen Universität
(TU) Berlin ein,\footnote{\url{http://bak-information.de/events/bak-09-19-zukunft-der-informationswissenschaft-hat-die-informationswissenschaft-eine-zukunft-eine-veranstaltung-des-bak-in-kooperation-mit-open-password/}
  {[}zuletzt gesehen am 14.10.2019{]}} um zusammen mit zwei
Professorinnen und einem Professor sowie zwei Masterstudierenden die
Diskussion bezüglich der Zukunft der Informationswissenschaft
fortzuführen.

Mehr als 60 Personen kamen in die Universitätsbibliothek der TU Berlin,
um sich an der Diskussion zur Zukunft der Informationswissenschaft zu
beteiligen. Nach der Begrüßung durch die Bibliotheksleitung sowie durch
die Vorsitzende des BAK Information, Tania Estler-Ziegler, hieß Willi
Bredemeier die Anwesenden willkommen. In seinem Statement erläuterte er,
welches Anliegen er mit dem von ihm herausgegebenen Sammelwerk verbinde.
Einerseits solle es eine Ermunterung zur Fortführung der
Grundsatzdebatte um die Zukunft der Informationswissenschaft sein,
andererseits eine Leistungsschau, die zeige, was in der
informationswissenschaftlichen Lehre sowie in der Forschung der
Informationswissenschaft aktuell geschehe. Nach der Vorstellung der
Referentinnen und Referenten des Abends übergab Willi Bredemeier das
Wort an Frauke Schade, Professorin für Informationsmarketing, PR und
Bestandsmanagement an der Hochschule für Angewandte Wissenschaften (HAW)
Hamburg und Vorsitzende der Konferenz der informations- und
bibliothekswissenschaftlichen Ausbildungs- und Studiengänge (KIBA).

Frauke Schade begann ihre Keynote mit dem Versuch einer Definition des
Felds der Informationswissenschaft. Sie definierte
Informationswissenschaft als interdisziplinäre Wissenschaft, die sich
als Handlungswissenschaft an den Bedarfen der Berufspraxis orientiere
und nach internationalem Begriffsverständnis alle kulturellen
Gedächtnisinstitutionen, also auch Bibliotheken, Archive und Museen,
einbeziehe. Anschließend präsentierte Schade die Ergebnisse einer
Auswertung, welche sie zusammen mit Klaus Gantert von der Hochschule für
den öffentlichen Dienst (HföD) in Bayern und Günther Neher von der
Fachhochschule (FH) Potsdam vorgenommen hatte. Gegenstand dieser
Auswertung waren Positionen und Strategien zur Digitalisierung aus der
Politik sowie deren Beratungsgremien, in denen sich Inhalte
informationswissenschaftlicher Forschung und Lehre verorten ließen. Im
Ergebnis habe sich gezeigt, dass informationswissenschaftliche
Kompetenzen zum Beispiel in den Bereichen Open Access und Open Science,
(Forschungs-)Datenmanagement, Digitale Langzeitarchivierung, Lern- und
Forschungsumgebungen, Informationsverhalten sowie Digitale Informations-
und Medienkompetenz dringend gebraucht würden. Die
Informationswissenschaft habe daher eindeutig eine Zukunft. Zu beklagen
sei lediglich ein Mangel an Fachpersonal, den Fachhochschulen und
Universitäten jedoch durch geeignete Kooperationen abmildern könnten,
beispielsweise indem sie Absolventinnen und Absolventen parallel zu
einer postgradualen Beschäftigung ein aufbauendes Studium ermöglichten,
so Schade abschließend.

Nach der Keynote von Frauke Schade hielt Dirk Lewandowski, Professor für
Information Research und Information Retrieval an der HAW Hamburg, sein
Impulsreferat mit dem Titel \enquote{Warum die Frage nach der Zukunft
der Informationswissenschaft falsch gestellt ist}. Hierin entkräftete
Lewandowski zunächst die drei Hauptkritikpunkte, die im Sammelwerk
\enquote{Zukunft der Informationswissenschaft} gegen die
Informationswissenschaft vorgebracht worden waren und dadurch Zweifel an
deren Zukunft(sfähigkeit) aufkommen ließen: Die fehlende Fundierung, die
fehlende Relevanz und der ungenügende Praxisbezug der
Informationswissenschaft. Danach erläuterte Lewandowski seine Sicht auf
die Informationswissenschaft. Für ihn gebe es, im Gegensatz zu vielen
Autoren des Sammelwerks, keine deutsche Informationswissenschaft,
sondern nur deutsche Informationswissenschaftlerinnen und
Informationswissenschaftler, die in internationale Kontexte eingebunden
seien. Alle erfolgreichen deutschen Informationswissenschaftler
arbeiteten international, würden dafür in Deutschland jedoch zu wenig
wahrgenommen. Um dies zu ändern schlug Lewandowski vor, die Community zu
stärken, die Leistungen der Informationswissenschaft stärker
herauszustellen, die Informationswissenschaft an Politik, Regulierung
und Gesellschaft anzubinden und Strategien zu entwickeln, wie die
Informationswissenschaft wachsen könne.

Vivien Petras, Professorin für Information Retrieval am und
geschäftsführende Institutsdirektorin des Instituts für Bibliotheks- und
Informationswissenschaft (IBI) der Humboldt-Universität (HU) zu Berlin,
sprach sich in ihrem Impulsreferat ebenfalls für eine Stärkung der
Community aus. Darüber hinaus argumentierte sie gegen eine Trennung von
Bibliotheks- und Informationswissenschaft, die sich so nur in
Deutschland entwickelt habe. International sei die
Bibliothekswissenschaft schon immer als Teilbereich einer breiteren
Informationswissenschaft begriffen worden. Dementsprechend sei es nur
folgerichtig, dass sich das IBI heute als ein
informationswissenschaftliches Institut verstehe, welches
Gedächtnisinstitutionen (allen voran Bibliotheken) als spezielle
Ausprägung von Informationsorganisationen untersuche.

Florian Dörr, Masterstudent am IBI der HU Berlin, beklagte in seinem
Impulsstatement den geringen Bekanntheitsgrad der
Informationswissenschaft. Ihr fehle die dringend benötigte Lobby in
Bevölkerung, Wissenschaft und Politik. Die Informationswissenschaft
könne jedoch enorm an Bedeutung gewinnen, wenn sie sich auf die
Vermittlung von Informationskompetenz fokussiere. Informationskompetenz
werde angesichts der gesellschaftlichen Herausforderungen, die mit dem
digitalen Wandel einhergingen, dringend gebraucht. Durch die
Fokussierung auf die Vermittlung von Informationskompetenz sowie durch
ein stärkeres politisches und gesellschaftliches Engagement von
Informationswissenschaftlerinnen und Informationswissenschaftlern könne
die Informationswissenschaft zum Wegbereiter der Wissensgesellschaft
werden.

Carmen Krause {[}Autorin dieses Beitrags{]}, Masterstudentin am
Fachbereich Informationswissenschaften der FH Potsdam, beantwortete die
Frage nach der Zukunft der Informationswissenschaft in ihrem
Impulsstatement mit drei kommentierten Gegenfragen: Von welchem
Zeitraum, welchem geografischen Bezugsraum und welcher
Informationswissenschaft ist eigentlich die Rede? Dabei wies sie auf den
zunehmenden Veränderungsdruck hin, der durch die digitale Transformation
entstehe, auf die fehlende Bekanntheit der Informationswissenschaft im
deutschsprachigen Raum sowie auf die Frage, ob es sich bei der
Informationswissenschaft überhaupt um eine akademische Disziplin
handelt. Letzteres verneinte Krause mit dem Hinweis auf das Fehlen eines
einheitlichen Selbstverständnisses sowie eigener Theorien und Methoden,
die sich auf einen bestimmten unikalen Gegenstandsbereich beziehen.
Daher, so Krause, sollte sich die Informationswissenschaft als
x-disziplinäres Forschungsfeld anstatt als akademische Disziplin
begreifen. Ferner schlug sie vor, das informationswissenschaftliche
Masterstudium beziehungsweise informationswissenschaftliche
Weiterbildungen am besten mit einem nicht-informationswissenschaftlichen
Bachelor zu kombinieren und Informationswissenschaft in Forschung und
Lehre standort- und disziplinübergreifend zu betreiben. So könnte die
Informationswissenschaft in Zukunft als x-disziplinäres Forschungsfeld
bezogen auf ganzheitliche Fragen die Information betreffend zwischen
Disziplinen vermitteln und übersetzen, und dadurch an Bedeutung sowie an
Sichtbarkeit gewinnen.

Nach diesem letzten Statement eröffnete die Moderatorin Michaela Jobb,
Leiterin der Bibliothek Wirtschaft und Management der TU Berlin und
Mitglied im BAK-Vorstand, die Podiumsdiskussion. Die erste Frage, ob sie
sich selbst als Informationswissenschaftler sähen, wurde von allen
Referentinnen und Referenten bejaht. Vivien Petras ergänzte, dass sie
sich in Deutschland manchmal auch explizit als Bibliotheks- und
Informationswissenschaftlerin bezeichne, um ihren Standpunkt zu
verdeutlichen. Elisabeth Simon, Inhaberin des Simon Verlags für
Bibliothekswissen, merkte an, dass man ihrer Erfahrung nach als
Informationswissenschaftler früher hohes Ansehen genossen, als
Bibliothekarin oder Bibliothekar hingegen kaum etwas gegolten habe.
Vielleicht, so spekulierte sie, habe sich die Trennung zwischen
Bibliotheks- und Informationswissenschaft in Deutschland auch deswegen
so hartnäckig halten können.

Auf die Frage, wie man der Zivilgesellschaft, der Politik und anderen
Entscheidungsträgern die Erkenntnisse und den Nutzen der
Informationswissenschaft näher bringen könne, verwies Dirk Lewandowski
nochmals auf die von ihm genannten Punkte. Zudem, antwortete
Lewandowski, müssten die Verbände wachsen und noch aktiver werden.
Handlungsaufträge an Einzelpersonen seien hingegen wirkungslos, da jeder
der Anwesenden bereits maximal viel für die Informationswissenschaft
tue. Tania Estler-Ziegler pflichtete Lewandowski bei und äußerte
ebenfalls, dass BAK, DGI, dbv und BIB ruhig noch aktiver werden könnten.
Rainer Kuhlen, ehemaliger Professor für Informationswissenschaft an der
Universität Konstanz, forderte Informationswissenschaftlerinnen und
Informationswissenschaftler auf, mehr zu publizieren. Helmut Voigt,
früherer Fachreferent an der Universitätsbibliothek der HU Berlin und Mitglied im BAK-Vorstand,
wandte dagegen ein, dass das Publizieren für Informationswissenschaftler
schwierig sei, da es sich bei der Informationswissenschaft, wie bei der
Statistik, eher um eine Methodenwissenschaft handele.

Jana Rumler, leitende Mitarbeiterin für bibliothekarische Dienste im
Leibniz-Zentrum für Agrarlandschaftsforschung und Vorstandsmitglied des
LIBREAS e.\,V. sowie Petra Schramm, Bibliothekarin in der Zentral- und
Landesbibliothek Berlin, brachten die Themen lebenslanges Lernen sowie
berufsbegleitende Aus- und Weiterbildung in die Diskussion ein.

Maxi Kindling, Mitarbeiterin im Open-Access-Büro Berlin und
Vorstandsmitglied des LIBREAS e.\,V., fragte, weshalb ausgerechnet ein
Sammelwerk zur Zukunft der Informationswissenschaft weder Open Access
noch digital erschienen sei. Alles andere als zukunftsweisend sei auch
die fehlende Diversität unter den Autorinnen und Autoren. Zur
Veranstaltung merkte sie an, dass der akademische Mittelbau die Zukunft
der Informationswissenschaft maßgeblich mitgestalten würde.
Dementsprechend wenig repräsentativ sei es, dass neben Professorinnen
und Professoren nur Studierende auf das Podium eingeladen worden seien.

Der Redebedarf über die Zukunft der Informationswissenschaft schien
insgesamt groß, die Beteiligung des Publikums war rege. Selbst nach der
Veranstaltung wurde teils heftig weiter diskutiert. Dabei zeigte sich
einmal mehr, wie schwer es der informationswissenschaftlichen Community
fällt, zu einem einheitlichen Verständnis von Informationswissenschaft
zu gelangen. Wenigstens konnte Einigkeit darin erzielt werden, dass die
Bibliothekswissenschaft auf jeden Fall zur Informationswissenschaft
dazugehöre und dass es eine wie auch immer geartete Zukunft für die
Informationswissenschaft geben werde.

%autor
\begin{center}\rule{0.5\linewidth}{0.5pt}\end{center}

Nach dem Abitur studierte \textbf{Carmen Krause} im Magisterstudiengang
Neuere und Neueste Geschichte sowie Neuere deutsche Literatur an der
Humboldt-Universität zu Berlin. Dort konnte sie als studentische
Beschäftigte der Zweigbibliothek Philosophie bereits während des
Studiums erste Berufserfahrungen im Bibliotheksbereich sammeln. Nach dem
Studium folgten Beschäftigungen in Bibliotheken von Unternehmen und
wissenschaftlichen Einrichtungen. Aus diesem Grund beschloss sie, ein
Studium im Bachelorstudiengang Bibliotheksmanagement an der
Fachhochschule Potsdam aufzunehmen, welches sie 2018 mit einer Arbeit
über die Potenziale des Internets der Dinge für Bibliotheken abschloss.
Für diese Arbeit wurde sie mit dem b.i.t.online Innovationspreis 2019
sowie mit dem Best Presentation Award des 9. Studierenden-Workshops für
informationswissenschaftliche Forschung ausgezeichnet. Derzeit studiert
sie im Masterstudiengang Informationswissenschaften an der
Fachhochschule Potsdam.

\end{document}
