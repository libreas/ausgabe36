\documentclass[a4paper,
fontsize=11pt,
%headings=small,
oneside,
numbers=noperiodatend,
parskip=half-,
bibliography=totoc,
final
]{scrartcl}

\usepackage[babel]{csquotes}
\usepackage{synttree}
\usepackage{graphicx}
\setkeys{Gin}{width=.4\textwidth} %default pics size

\graphicspath{{./plots/}}
\usepackage[ngerman]{babel}
\usepackage[T1]{fontenc}
%\usepackage{amsmath}
\usepackage[utf8x]{inputenc}
\usepackage [hyphens]{url}
\usepackage{booktabs} 
\usepackage[left=2.4cm,right=2.4cm,top=2.3cm,bottom=2cm,includeheadfoot]{geometry}
\usepackage{eurosym}
\usepackage{multirow}
\usepackage[ngerman]{varioref}
\setcapindent{1em}
\renewcommand{\labelitemi}{--}
\usepackage{paralist}
\usepackage{pdfpages}
\usepackage{lscape}
\usepackage{float}
\usepackage{acronym}
\usepackage{eurosym}
\usepackage{longtable,lscape}
\usepackage{mathpazo}
\usepackage[normalem]{ulem} %emphasize weiterhin kursiv
\usepackage[flushmargin,ragged]{footmisc} % left align footnote
\usepackage{ccicons} 
\setcapindent{0pt} % no indentation in captions

%%%% fancy LIBREAS URL color 
\usepackage{xcolor}
\definecolor{libreas}{RGB}{112,0,0}

\usepackage{listings}

\urlstyle{same}  % don't use monospace font for urls

\usepackage[fleqn]{amsmath}

%adjust fontsize for part

\usepackage{sectsty}
\partfont{\large}

%Das BibTeX-Zeichen mit \BibTeX setzen:
\def\symbol#1{\char #1\relax}
\def\bsl{{\tt\symbol{'134}}}
\def\BibTeX{{\rm B\kern-.05em{\sc i\kern-.025em b}\kern-.08em
    T\kern-.1667em\lower.7ex\hbox{E}\kern-.125emX}}

\usepackage{fancyhdr}
\fancyhf{}
\pagestyle{fancyplain}
\fancyhead[R]{\thepage}

% make sure bookmarks are created eventough sections are not numbered!
% uncommend if sections are numbered (bookmarks created by default)
\makeatletter
\renewcommand\@seccntformat[1]{}
\makeatother

% typo setup
\clubpenalty = 10000
\widowpenalty = 10000
\displaywidowpenalty = 10000

\usepackage{hyperxmp}
\usepackage[colorlinks, linkcolor=black,citecolor=black, urlcolor=libreas,
breaklinks= true,bookmarks=true,bookmarksopen=true]{hyperref}
\usepackage{breakurl}

%meta

%meta

\fancyhead[L]{P. Kampa \\ %author
LIBREAS. Library Ideas, 36 (2019). % journal, issue, volume.
\href{http://nbn-resolving.de/}
{}} % urn 
% recommended use
%\href{http://nbn-resolving.de/}{\color{black}{urn:nbn:de...}}
\fancyhead[R]{\thepage} %page number
\fancyfoot[L] {\ccLogo \ccAttribution\ \href{https://creativecommons.org/licenses/by/4.0/}{\color{black}Creative Commons BY 4.0}}  %licence
\fancyfoot[R] {ISSN: 1860-7950}

\title{\LARGE{User-centering and constant communication. Lessons learnt from the Open Repositories 2019 in Hamburg, June 10–13. A short congress report}} % title
\author{Philipp Kampa} % author

\setcounter{page}{1}

\hypersetup{%
      pdftitle={User-centering and constant communication. Lessons learnt from the Open Repositories 2019 in Hamburg, June 10–13. A short congress report},
      pdfauthor={Philipp Kampa},
      pdfcopyright={CC BY 4.0 International},
      pdfsubject={LIBREAS. Library Ideas, 36 (2019).},
      pdfkeywords={Open Repositories 2019, Kongressbericht, Forschungsinfrastruktur, Repositorien, user experience},
      pdflicenseurl={https://creativecommons.org/licenses/by/4.0/},
      pdfcontacturl={http://libreas.eu},
      baseurl={http://libreas.eu},
      pdflang={en},
      pdfmetalang={de}
     }



\date{}
\begin{document}

\maketitle
\thispagestyle{fancyplain} 

%abstracts

%body
\emph{Open Repositories} (\emph{OR}) is an international conference
dating back to the Noughties. \emph{OR 2019} put \emph{users} with their
perspectives and their possible impacts on infrastructures at the
forefront.\footnote{In their talk on \enquote{Jisc Open Research
  Repository: Delivering a compelling User Experience}, Davey, Fripp \&
  Kaye {[}Jisc, UK{]} expressed the following: \enquote{The repository
  isn't the need {[}...{]} the need is the thing the repository does.
  The repository is kind of the symptom of a whole set of needs that lie
  outside of the repository.} (Davey, Fripp \& Kaye, 11 June 2019,
  \url{https://lecture2go.uni-hamburg.de/l2go/-/get/v/24954}, approx.
  9:55-10:05, last retrieved 6 August 2019). See also Jeff Gothelf in
  his keynote \enquote{Outcomes over output: A user-centric approach to
  building successful systems}: \enquote{Just because you can build it,
  doesn't mean you should build it.} (Gothelf, 11 June 2019,
  \url{https://lecture2go.uni-hamburg.de/l2go/-/get/v/24934}, approx.
  44:06, last retrieved 6 August 2019) ; this is by the way, the aspect
  that his differentiation (cf.~Gothelf/Seiden 2016: 34 and 21:
  \enquote{Our goal is not to create a deliverable or a feature: it's to
  positively affect customer behavior or change in the world---to create
  an outcome.}) between output (just build something) and outcome (build
  something causing a real, measurable change) boils down to.} The
conference ran under the motto \enquote{All the user needs}. It focused
on the reception of and familiarity with open repositories by people or
even machines.\footnote{Cf.
  http://archiv.gwin.gwiss.uni-hamburg.de/or2019/cfp/, last retrieved 12
  December 2019. Hicks, Phillipps \& Andrews {[}University of North
  Texas Libraries, USA{]} (the talk was held by William Hicks), 11 June
  2019, \url{https://lecture2go.uni-hamburg.de/l2go/-/get/v/24757}, last
  retrieved 8 August 2019 covered this aspect by showing which
  \enquote{Links for Robots} (17:35, on the slide blended in) they
  provided in their repository; Hicks, Phillipps \& Andrews referred to
  \enquote{IIIF-manifestations of the objects} (17:45), for example.}
Accordingly, users and their needs and behaviour were the topics running
through the presentations. As stated, \emph{OR} as a conference format
dates back more than a decade. This not only indicates a long tradition,
but also the stability and solidity of the topic itself: Open
repositories have obviously become an integral part of scholarship, as
well as of library and information systems.

The conference lasted for four days. It was divided into a workshop day
and three panel days, the latter consisting each of two panel sections,
separated into five tracks running parallel. Taking the broad spectrum
and the overlapping structure into consideration, this report does not
aim to describe the conference and its results in an exhaustive
way.\footnote{By now, most of the content of Open Repositories 2019 is
  available online, cf.
  \url{https://lecture2go.uni-hamburg.de/l2go/-/get/l/5134}, last
  retrieved 5 August 2019. It can also be found in a Zenodo community
  (not yet available/under construction, cf.~for current information:
  \url{http://archiv.gwin.gwiss.uni-hamburg.de/or2019/}, last retrieved
  12 December2019); long-term archiving should therefore be ensured. The
  recorded material is referred to in the text at hand, by naming the
  authors, and, if applicable, citing the corresponding part of the
  video in minutes and seconds. Affiliations of the contributors,
  referred to in this text, are noted down in square brackets;
  specifications follow the data available on \emph{OR2019}'s conference
  agenda (cf. \url{https://www.conftool.net/or2019/sessions.php}, last
  retrieved 10 September 2019), written down here in the report solely
  at first mention.} Rather it presents a series of snapshots focusing
on the sustainability of structures, which was one of the themes
addressed at the conference. Soft factors like user-centric approaches,
as well as constant external and internal communication are keys to
sustainable (infra-)structures for open repositories.

The conference started with a workshop day, followed by an opening
keynote (Gothelf), several panels, poster and ideas challenge
presentations and a closing keynote (Piwowar). In accordance to the
description on the conference's website,\footnote{\enquote{The annual
  Open Repositories Conference is a practitioner based conference that
  brings together people from higher education, government, libraries,
  archives and museums to focus on repository infrastructure, tools,
  services, and policies.}
  \url{http://archiv.gwin.gwiss.uni-hamburg.de/or2019/23/}, last
  retrieved 12 December 2019.} repository \emph{practice} stood
centre-stage. Not surprisingly, numerous projects (in different stages)
facilitating several approaches were presented and discussed throughout
the four conference days -- from a very deep and detailed user-centered
design (Galligan {[}Rockefeller Archive Center, USA{]}), to remarks on
institutional issues one should be aware of when it comes to forming
sustainable organisational patterns (Notay, Moore \& Duke {[}Jisc,
UK{]}). The presentations were not only wide-ranging in terms of the
\enquote{locations}, and the methods shown, but also concerning the
relevant topics such as presenting scientific audio material in a
repository (Plank, Drees \& Ogunyemi {[}Leibniz Information Centre for
Science and Technology, Germany{]}), providing scientific images
(Sohmen, Blümel \& Heller {[}Leibniz Information Centre for Science and
Technology, Germany{]}), and preserving (all around) country music
(Boulie {[}Country Music Hall of Fame and Museum, USA). In short, open
repositories were not only discussed in relation to text items, nor was
the focus solely on the academic sector, but also considered cultural
heritage.

Many of the talks reflected the user-centric approach by sharing the
results of research into user needs. Methods applied were: building
focus groups, doing expert interviews, surveys, testing, and shaping
personas (e.g.~Géroudet {[}Université de Lille, France{]}; Plank, Drees
\& Ogunyemi). Applying a mixture of these methods to create helpful
products seems to be at the core of repositories' efforts on behalf of
users. Further, Davey, Fripp \& Kaye underlined the importance of
navigating and supporting users through the \enquote{journey}\footnote{Davey,
  Fripp \& Kaye, 11 June 2019,
  \url{https://lecture2go.uni-hamburg.de/l2go/-/get/v/24954}, approx.
  05:21, last retrieved 7 August 2019. Another aspect mentioned was
  accessibility. Not only Davey, Fripp \& Kaye referred to this topic,
  but also Hicks, Phillipps \& Andrews, 11 June 2019,
  \url{https://lecture2go.uni-hamburg.de/l2go/-/get/v/24757}, last
  retrieved 7 August 2019, presenting the processes undergone in terms
  of a redesign of the University of North Texas Libraries' repository,
  cf.~especially 07:20-07:40.} they go on, for example when depositing
something, as well as general design principles such as the appropriate
harmonisation of colours.

Also of interest were approaches that focused on the sustainability of
structures, such as the management of persistent identifiers (PIDs)
being a key element of repositories. Notay, Moore \& Duke's talk (the
presentation was held by Balviar Notay and Monica Duke) about building
an ORCID consortium in the UK included a \enquote{business plan for
sustainability}.\footnote{Notay, Moore \& Duke, 12 June 2019,
  \url{https://lecture2go.uni-hamburg.de/l2go/-/get/v/24860}, last
  retrieved, 2 August 2019, approx. 07:23.} Sustainability was drilled
down to 1) \enquote{developing infrastructure}, 2) \enquote{maintaining
integration}, 3) \enquote{engaging the community}, 4)
\enquote{developing the business model}, 5) \enquote{communications
planning}, 6) \enquote{gathering and synthesising the requirements} and
7) \enquote{international engagement}.\footnote{Ibid., approx.
  26:55-27:29.} Notay, Moore \& Duke linked (added) value, and
communication with sustainability. Consequently, they underlined, that
the question of costs is not the only matter to consider when planning
wider institutional structures, such as a consortium.\footnote{Cf.
  ibid., approx. 23:48-23:58. In this context, it could be added, that
  within the panel \enquote{All Together Now? The role of service
  providers in open source communities} (cf.
  \url{https://lecture2go.uni-hamburg.de/l2go/-/get/v/24920}, last
  retrieved, 6 August 2019, the question of the need of commercial
  stakeholders in the field of open repositories was raised: Mark Bussey
  from Digital Curation Experts (DCE) mentioned in his contribution to
  that issue: 1) Commercial providers do work together with a range of
  clients, so they are aware of the variety of challenges. 2) Commercial
  providers facilitate sustainability in a community whose institutions
  are in constant transition, he talked about \enquote{gaps} (07:33)
  that naturally come into being, for example concerning focusses and
  training. 3) Commercial providers decrease obstacles small
  institutions may face when trying to participate in the open
  repository world (cf.~for all approx. 06:40-07:50). Richard Jones from
  Cottage Labs added that except from that, the division between
  commercial and non-commercial providers within the open repository
  world seems to be \enquote{artificial} because finally, it is about
  this: \enquote{serve ourselves by serving others} (approx. 10:36).}
Therefore, they worked with the metaphorical phrase
\enquote{Sustainability requires spinning plates}.\footnote{Notay, Moore
  \& Duke, 12 June 2019,
  \url{https://lecture2go.uni-hamburg.de/l2go/-/get/v/24860}, last
  retrieved, 9 August 2019, approx. 27:13, slide blended in.} In
addition, they raised the issue of PID interoperability being important
for sustainability. The challenge is to find a balance between the
diversification and the association of structures.

Adaptability to the ever-changing user behaviour was also addressed in
this context. Lee Boulie presented a project on Music Row, a part of
Nashville in the US that, with many music studios located there, has
played a decisive role in music history. Recent efforts from the Country
Music Hall of Fame and Museum were summed up with the phrase
\enquote{Meet Them Where They Are}.\footnote{\url{https://www.conftool.net/or2019/index.php?page=browseSessions&form_session=353} = \url{https://www.conftool.net/or2019/index.php/24x7-P3C-108Boulie_b.pptx?page=downloadPaper\&filename=24x7-P3C-108Boulie_b.pptx\&form_id=108\&form_index=2\&form_version=final}, last retrieved 7 August 2019, slides 8 \& 9.} According to Boulie,
access points in cultural heritage institutions tend to be apps and
video games, while \enquote{classical anchors}, like institutional
homepages, are losing significance. As the conference's motto already
reflects, and as Boulie's talk proved, users are at the core of open
repositories. This statement seems to be true for users in terms of
readers and authors as well as library services: As Zhang et
al.~{[}Oregon State University Libraries and Press{]} showed, it is a
challenge to foster openness by increasing green open access deposit
rates, without also adding to the workload of library staff who must
process and accompany these deposits. Oregon State developed and
implemented an automated system (based on Web of Science), which is
called Easy Deposit 2 (ED2). This system sends out emails inviting
deposit, containing a short text with links to click on. The process of
depositing is reduced to uploading a file, and then confirming the
publication on ScholarsArchive@OSU. Authors do not need an account in
order to contribute to the repository. Zhang et al.~however, emphasised
that fostering openness in a manifest and sustainable way depends on
research culture and the users' attitudes. One of the points also named,
was the importance of statistics about items in the
repositories.\footnote{Cf. Zhang et al., 12 June 2019,
  \url{https://lecture2go.uni-hamburg.de/l2go/-/get/v/24853}, last
  retrieved 7 August 2019.} Contributors are highly interested in
download and view rates, a factor that seems to reflect the like/dislike
and rating-behaviour to be observed in everyday life.

Furthermore, a wide range of repositories was presented, e.g.~scientific
videos in an open repository hosted by TIB Hannover (Plank et
al.)\footnote{Cf. Plan, Drees \& Ogunyemi, 11 June 2019,
  \url{https://lecture2go.uni-hamburg.de/l2go/-/get/v/24777}, last
  retrieved 6 August 2019.}, Vanhaverbeke and colleagues {[}KU Leuven,
Belgium{]} presented University of Leuven's new repository LIRIAS
2.0\footnote{The talk was held by Valérie Adriaens.} to show that the
extension of formats and item types, such as \enquote{datasets, software
\& code},\footnote{\url{httpswww.conftool.netor2019index.phppagebrowsesessionsform_session377presentationsshow} = \url{https://www.conftool.net/or2019/index.php/24x7-P7C-497Adriaens_a.pdf?page=downloadPaper\&filename=24x7-P7C-497Adriaens_a.pdf\&form_id=497\&form_version=final},
  last retrieved 14 November 2019, slide/page 10.} is an important
factor.\footnote{Cf. Vanhaverbeke et al., 13 June 2019,
  \url{https://lecture2go.uni-hamburg.de/l2go/-/get/v/24900}, last
  retrieved 7 August 2019.} Vanhaverbeke et al.~also stated in their
talk that \enquote{communication was really crucial}.\footnote{Ibid,
  approx. 07:15-07:17.} Constant communication does not only include
asking and finding out what users need, but also monitoring their
behaviour. A workshop on algorithmic awareness, held on the first
conference day (Clark \& Kaptanian {[}Montana State University, USA{]}),
fitted well into this topic : The workshop raised the question of how to
accompany and support users in a digital world, and it sensitised those
on the providers' side, including librarians and developers, to the
power of infrastructural decisions. Openness demands transparency and
this involves algorithmic awareness being a part of information
literacy.\footnote{Cf. for material:
  \url{https://github.com/jasonclark/algorithmic-awareness}, last
  retrieved 9 August 2019.}

The conference was framed by two keynotes, the first held by Jeff
Gothelf, a freelance coach and consultant,\footnote{Cf.
  \url{https://jeffgothelf.com/}, last retrieved 3 November 2019.} the
second by Heather Piwowar, co-founder of \enquote{our research},
formerly ImpactStory, the organisation behind Unpaywall.\footnote{Cf.
  \url{https://ourresearch.org/team}, last retrieved 3 November 2019.}
Jeff Gothelf had a look at design principles, managerial decisions
(keywords: design thinking and agile methods), and workflow models.
Heather Piwowar talked about the growth and significance of the open
repositories' community in general. Jeff Gothelf underlined, that
\enquote{we live in a world where the dominant tech companies create the
tools that we use every day {[}...{]} and those expectations come home
to the systems that we build}.\footnote{Gothelf, 11 June 2019,
  \url{https://lecture2go.uni-hamburg.de/l2go/-/get/v/24934}, approx.
  03:37-03:56, last retrieved 6 August 2019.} Piwowar's talks turned out
to be a passionate speech for openness concerning software, but also in
terms of the users and their needs. She said the memorable sentences
\enquote{think of your users most broadly},\footnote{Piwowar, 13 June
  2019, \url{https://lecture2go.uni-hamburg.de/l2go/-/get/v/24979},
  approx. 38:25, last retrieved 6 September 2019.} and \enquote{think
about all the users you haven't met yet}.\footnote{Ibid., approx.
  38:39-38:42.} Furthermore, she mentioned openness of the community by
passing the knowledge gained over time.\footnote{Cf. ibid., approx.
  24:00-25:30.} In this way, Heather Piwowar's talk might be linked to
the keynote held by Jeff Gothelf -- and to his works on UX in general:
By defining UX as \enquote{the sum total of all the interactions a user
has with {[}a{]} product and service} (Gothelf/Seiden 2016: 47),
openness seems to be on top of this process, as users and their needs
are, like the world itself, flowing. To find out what user experience
is, it is necessary to keep in touch with users and to constantly
communicate with them. Openness is appropriate to foster sustainable
structures in libraries.

To sum it up: With its focus on \enquote{All the user needs}, 2019's
\emph{Open Repositories} covered a topic that affects all aspects of
open repositories. One cannot divide the user's needs from the
provider's, and the institutional perspective remains connected to the
products and services. Having a closer look at the question of
sustainability, lessons learnt from \emph{OR2019} appear to be:
sustainability can only be guaranteed by centering all efforts on the
users and their needs. This in turn requires constant communication with
users, as well as within teams and across institutions and communities,
and all those providing products and services. \emph{OR2019}, with its
long tradition of international exchange and openness itself is a good
example of such efforts.\footnote{\emph{OR2020} will take place in
  Stellenbosch, South Africa, cf. \url{https://or2020.sun.ac.za}, last
  retrieved 4 September 19.}

References:

Gothelf, Jeff/Seiden, Josh (2016): Lean UX. Designing Great Products
with Agile Teams. 2\textsuperscript{nd} Edition. O'Reilly.

Acknowledgements:

The author would like to thank LIBREAS e.V. for generously supporting
the congress' attendance, as well as for beneficially commenting on this
report. He would further like to thank Benjamin Auberer for helpful
remarks and notes on the text at hand; many thanks also to Universitäts-
und Landesbibliothek Sachsen-Anhalt, Halle (Saale) for supporting me.

%autor
\begin{center}\rule{0.5\linewidth}{\linethickness}\end{center}

\textbf{Philipp Kampa} is working as a trainee at the University and
State Library of Saxony-Anhalt, Halle (Saale). As part of his
traineeship, he is completing the distance learning course ``Master of
Arts (Library and Information Science) M. A. (LIS)''. Before it, he
worked as a research assistant at the Martin Luther University
Halle-Wittenberg.

\end{document}
