In diesem Essay wird eine radikale Lösung des Problems wirklich
langfristiger Datensicherung und -verfügbarkeit vorgeschlagen. Sie
beruht auf der zuvor erläuterten Einschätzung, dass die Vielfalt und
Komplexität heutiger IT-Systeme weder mit vertretbarem Aufwand für die
Zukunft konserviert werden kann, noch in der Zukunft rekonstruierbar
sein wird. Ein Ausweg könnte die Schaffung einer (internationalen)
Institution mit ``Ewigkeitsgarantie'' sein, welche die Entwicklung und
Betreuung einer freien Hard- und Softwareplattform übernimmt, mit der
insbesondere Forschungsprojekte und -einrichtungen, aber auch Firmen
oder Privatpersonen arbeiten können, die ihre Daten abschließend der
Institution zur dauerhaften Speicherung übergeben wollen. Bei der
Entwicklung dieser Plattform könnten vorausweisende Lösungen aus der
bisherigen IT-Enwicklung übernommen; bekannte, aber aus
Kompatibilitätsgründen nicht behobene Fehler und Irrtümer jedoch
vermieden werden. Außerdem könnte das System zugleich als eine wirklich
offene Plattform für Repositorien zur Veröffentlichung von Daten und
Ergebnissen im Open Access bilden und undurchsichtige Verfahren
beispielsweise im Peer Review und der Messung von Forschungsrelevanz
ablösen. Da Bibliotheken, Archive und Museen diejenigen Institutionen
sind, die über umfangreichste historisch gewachsene Erfahrungen in der
Speicherung und Erschließung von Daten im weitesten Sinne haben, sollten
sie bei Konzeption und Betrieb dieser Institution eine führende Rolle
spielen.
