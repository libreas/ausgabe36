\documentclass[a4paper,
fontsize=11pt,
%headings=small,
oneside,
numbers=noperiodatend,
parskip=half-,
bibliography=totoc,
final
]{scrartcl}

\usepackage[babel]{csquotes}
\usepackage{synttree}
\usepackage{graphicx}
\setkeys{Gin}{width=.4\textwidth} %default pics size

\graphicspath{{./plots/}}
\usepackage[ngerman]{babel}
\usepackage[T1]{fontenc}
%\usepackage{amsmath}
\usepackage[utf8x]{inputenc}
\usepackage [hyphens]{url}
\usepackage{booktabs} 
\usepackage[left=2.4cm,right=2.4cm,top=2.3cm,bottom=2cm,includeheadfoot]{geometry}
\usepackage{eurosym}
\usepackage{multirow}
\usepackage[ngerman]{varioref}
\setcapindent{1em}
\renewcommand{\labelitemi}{--}
\usepackage{paralist}
\usepackage{pdfpages}
\usepackage{lscape}
\usepackage{float}
\usepackage{acronym}
\usepackage{eurosym}
\usepackage{longtable,lscape}
\usepackage{mathpazo}
\usepackage[normalem]{ulem} %emphasize weiterhin kursiv
\usepackage[flushmargin,ragged]{footmisc} % left align footnote
\usepackage{ccicons} 
\setcapindent{0pt} % no indentation in captions

%%%% fancy LIBREAS URL color 
\usepackage{xcolor}
\definecolor{libreas}{RGB}{112,0,0}

\usepackage{listings}

\urlstyle{same}  % don't use monospace font for urls

\usepackage[fleqn]{amsmath}

%adjust fontsize for part

\usepackage{sectsty}
\partfont{\large}

%Das BibTeX-Zeichen mit \BibTeX setzen:
\def\symbol#1{\char #1\relax}
\def\bsl{{\tt\symbol{'134}}}
\def\BibTeX{{\rm B\kern-.05em{\sc i\kern-.025em b}\kern-.08em
    T\kern-.1667em\lower.7ex\hbox{E}\kern-.125emX}}

\usepackage{fancyhdr}
\fancyhf{}
\pagestyle{fancyplain}
\fancyhead[R]{\thepage}

% make sure bookmarks are created eventough sections are not numbered!
% uncommend if sections are numbered (bookmarks created by default)
\makeatletter
\renewcommand\@seccntformat[1]{}
\makeatother

% typo setup
\clubpenalty = 10000
\widowpenalty = 10000
\displaywidowpenalty = 10000

\usepackage{hyperxmp}
\usepackage[colorlinks, linkcolor=black,citecolor=black, urlcolor=libreas,
breaklinks= true,bookmarks=true,bookmarksopen=true]{hyperref}
\usepackage{breakurl}

%meta
\appto\UrlBreaks{\do\a\do\b\do\c\do\d\do\e\do\f\do\g\do\h\do\i\do\j
\do\k\do\l\do\m\do\n\do\o\do\p\do\q\do\r\do\s\do\t\do\u\do\v\do\w
\do\x\do\y\do\z}

%meta

\fancyhead[L]{F. Gelati \\ %author
LIBREAS. Library Ideas, 36 (2019). % journal, issue, volume.
\href{http://nbn-resolving.de/}
{}} % urn 
% recommended use
%\href{http://nbn-resolving.de/}{\color{black}{urn:nbn:de...}}
\fancyhead[R]{\thepage} %page number
\fancyfoot[L] {\ccLogo \ccAttribution\ \href{https://creativecommons.org/licenses/by/4.0/}{\color{black}Creative Commons BY 4.0}}  %licence
\fancyfoot[R] {ISSN: 1860-7950}

\title{\LARGE{Die nachhaltige Bewahrung einer Forschungsdatenbank durch Linked Data. Laut welchem Vokabular?}} % title
\author{Francesco Gelati} % author

\setcounter{page}{1}

\hypersetup{%
      pdftitle={Die nachhaltige Bewahrung einer Forschungsdatenbank durch Linked Data. Laut welchem Vokabular?},
      pdfauthor={Francesco Gelati},
      pdfcopyright={CC BY 4.0 International},
      pdfsubject={LIBREAS. Library Ideas, 36 (2019).},
      pdfkeywords={Linked Data, Forschungsdaten, Standardisierung, Vokabulare, Wikidata},
      pdflicenseurl={https://creativecommons.org/licenses/by/4.0/},
      pdfcontacturl={http://libreas.eu},
      baseurl={http://libreas.eu},
      pdflang={en},
      pdfmetalang={en}
     }




\date{}
\begin{document}

\maketitle
\thispagestyle{fancyplain} 

%abstracts
\begin{abstract}
\noindent
Linked Data bietet die Möglichkeit nicht nur normkonforme Datensätze zur
Verfügung zu stellen sonder auch nachhaltig und langfristig Metadaten zu
bewahren, wenn diese zu Linked Data Vokabularen verknüpft sind. Aber
welches Vokabular soll man nutzen? Der Anwendungsfall einer
Forschungsdatenbank des Instituts für Zeitgeschichte München - Berlin
zeigt die Vorteile, Wikidata für die Erstellung von
Linked-Data-Beständen zu benutzen und Wikidata/DBpedia als
Backup-Vokabular zu verwenden.
\end{abstract}

%body
\hypertarget{normgerechte-metadaten-als-linked-data}{%
\section{Normgerechte Metadaten als Linked
Data}\label{normgerechte-metadaten-als-linked-data}}

Nur normgerechte und FAIR\footnote{FAIR Principles:
  \url{https://www.go-fair.org/fair-principles/}.}-Metadaten sind wert,
langfristig gespeichert zu werden. Zugleich erscheinen aber immer wieder
neue Standards! Das bedeutet, dass aktuell normkonforme Metadaten in der
Zukunft veraltet und kaum wiederverwendbar sein können. Das
Dublin-Core-Datenformat, das im ersten Jahrzehnt des 21. Jahrhunderts
sehr erfolgreich war, wird beispielsweise in Europa bereits heute immer
seltener genutzt, da sich das Europeana-Datenmodell durchsetzt. Mit
Linked Data kann man Datensätze erstellen, die später flexibel quasi als
Informationsatome auch nach neuen Standards verknüpft und
weiterverarbeitet werden können.

Das \emph{Institut für Zeitgeschichte München - Berlin}\footnote{\url{https://www.ifz-muenchen.de/}}
betreibt eine Forschungsdatenbank von personenbezogenen Daten mit mehr
als 3000 Normdaten (Authority Records), die aus der proprietären
Software für Sammlungsmanagement\footnote{FAUST (c) by Land
  Software-Entwicklung} als XML-Bestände exportiert werden können. Damit
die Daten nachhaltig bewahrt werden, arbeitet das Institut an der
Transformation der Einträge in Linked Data (Gelati 2019). Dafür wurde
bereits eine Mapping-Musterdatei getestet, sodass man in den
Quelldateien relevante Informationen selektieren und sie in die
geeigneten Felder der Zieldateien überführen kann.

Da Linked-Data-konforme N-Triples-Bestände aus
Subjekt-Prädikat-Objekt-Sätzen bestehen, ist es erforderlich zu
bestimmen, welche Prädikate von welchen Linked Data-Vokabularen benutzt
werden sollen.

Mit Wikidata, dem Normdaten-Projekt der Wikimedia Foundation, kann man
beispielsweise den folgenden Satz erstellen:

\begin{verbatim}
http://example.com/33
https://www.wikidata.org/wiki/Property:P20
https://www.wikidata.org/wiki/Q3004.
\end{verbatim}

Dieser Satz bedeutet: Jene Person, die über den persistenten Identifier
\texttt{example.com/33} identifiziert wird, hat Ingolstadt als
Sterbeort.

\hypertarget{linked-data-vokabulare}{%
\section{Linked Data Vokabulare}\label{linked-data-vokabulare}}

Doch warum soll man sich für Wikidata entscheiden? Das Repositorium
\url{https://lov.linkeddata.es/} weist mehr als 100
Linked-Data-Vokabulare nach, die oft gleichwertige Einträge haben.
Lassen wir uns das vorgenannte N-Triple-Beispiel weiter untersuchen.
Mehrere Vokabulare enthalten die Eigenschaft \enquote{Sterbeort}:

\begin{verbatim}
https://www.wikidata.org/wiki/Property:P20

https://d-nb.info/standards/elementset/gnd#placeOfDeath

http://id.loc.gov/ontologies/madsrdf/v1.html#deathPlace

http://dbpedia.org/ontology/deathPlace

http://sparql.cwrc.ca/ontology/cwrc.html#hasDeathPlace

http://www.rdaregistry.info/Elements/u/#P60592

https://schema.org/deathPlace
\end{verbatim}

Jedes dieser Vokabulare kann als der zweite Teil des Beispielsatzes
(statt \url{https://www.wikidata.org/wiki/Property:P20}) Anwendung
finden, um gleichlautende Resultate zu erzielen.

Flexibilität gibt es auch mit den Objekten: Lassen wir uns das
Sterbedatum prüfen. Meines Wissens bietet nur Wikidata eine exakte
Entität für das Jahr 1920. Aber das Jahr 1920 kann sowohl als die volle
Entität als auch als Kombination von

\texttt{integer("1920")} und
\texttt{class(https://www.wikidata.org/wiki/Q577)}

ausgedrückt werden.

\begin{verbatim}
http://example.com/33
https://www.wikidata.org/wiki/Property:P570
https://www.wikidata.org/wiki/Q2155 .

http://example.com/33
https://www.wikidata.org/wiki/Property:P570
"1920"^^https://www.wikidata.org/wiki/Q577 .
\end{verbatim}

Um einen Kalendertag auszudrücken, bleibt nur die letztere Option, denn
Wikidata bietet beispielsweise für den Tag \enquote{04.01.1988} keinen
eigenen persistenten Identifier.

\begin{verbatim}
http://example.com/33
https://www.wikidata.org/wiki/Property:P570
"04.01.1988"^^https://www.wikidata.org/wiki/Q205892 .
\end{verbatim}

Einrichtungen, die Linked Data generieren, können ihr eigenes Vokabular
entwickeln sowie bereits bestehende Ontologien nutzen (Guernaccini,
Mazzini, Bruno 2019). Die meisten arbeiten mit beiden Optionen: Die
Deutsche Nationalbibliothek\footnote{Ich nehme als Beispiel die Datei
  \url{https://d-nb.info/129307343/about/lds.rdf}.} erstellt Linked Data
sowohl mit ihrer eigenen \enquote{GND Ontology}\footnote{GND Ontology:
  \url{https://d-nb.info/standards/elementset/gnd}.} als auch mit den
berühmten \enquote{FOAF (Friend Of A Friend) Vocabulary
Specification}\footnote{Friend of a Friend (FOAF):
  \url{http://www.foaf-project.org/}.} und \enquote{Dublin Core Metadata
Terms}\footnote{Dublin Core Metadata Initiative: DCMI Metadata Terms:
  \url{https://www.dublincore.org/specifications/dublin-core/dcmi-terms/}.}.

Sind der große Erfolg im deutschsprachigen Raum der Gemeinsamen
Normdatei\footnote{Gemeinsame Normdatei (GND):
  \url{https://www.dnb.de/gnd}.} (aber momentan nicht der GND Ontology)
sowie die weltweite Nutzung der oben genannten Vokabulare ein
ausreichender Grund dafür, sie heute als nachhaltige Lösung für die
Langzeitarchivierung zu schätzen? Wenn derselbe Satz mit verschiedenen
Vokabularen zweimal formuliert würde, würde das unerwartete Ende der
Weiterpflege (oder Migration, oder Änderung) eines Vokabulars keinen
Informationsverlust verursachen.

\hypertarget{wikidata-als-linked-data-mittelpunkt}{%
\section{Wikidata als Linked
Data-Mittelpunkt}\label{wikidata-als-linked-data-mittelpunkt}}

Bedeutet die nachhaltige Datenarchivierung von Linked Data nicht, wenn
möglich, immer zwei Vokabulare zu nutzen? Das heißt, ein
\enquote{Backup-Vokabular} für den Fall fragmentierter, instabiler und
wechselhafter Ontologien zu haben? Wenn ja, welches?

Dafür scheint Wikidata die Lösung zu sein. Mehrere wissenschaftliche
Beiträge haben \enquote{Wikidata as a linking hub} (Neubert 2017) oder
\enquote{Wikidata as a universal identifier} (van Veen 2019) bereits
vorgeschlagen. Wikidatas reiches Angebot von sowohl Entitäten als auch
Eigenschaften ermöglicht es, Wikidata und die nahestehende Datenbank
DBpedia\footnote{DBpedia: \url{https://wiki.dbpedia.org/}.} als
\enquote{Backup-Vokabular} bzw. \enquote{Linked Data hub} zu verwenden.
Genau dieser Weg wird fortan vom Institut für Zeitgeschichte München -
Berlin mit seiner Forschungsdatenbank von personenbezogenen Daten
getestet.

Derselbe Salz wird dank XSL-Mapping einmal laut des fachspezifischen
(archivalischen) Standards RiC (Records in Contexts Ontology)

\begin{verbatim}
http://example.com/33
http://purl.org/ica/ric#RiC-hadDeathDate
"04.01.1988"^^https://schema.org/Date .
\end{verbatim}

und einmal laut der fachübergreifenden Datenbank Wikidata erscheinen.

\begin{verbatim}
http://example.com/33
https://www.wikidata.org/wiki/Property:P570
"04.01.1988"^^https://www.wikidata.org/wiki/Q205892 .
\end{verbatim}

Wäre es für tausende Einträge redundant oder komplementär? Übermäßig
oder vorsichtig? Arbeitsintensiv oder nachhaltig? Hinweise und
Kommentare sind herzlich willkommen.

\hypertarget{bibliographie}{%
\section{Bibliographie}\label{bibliographie}}

Francesco Gelati. (2019). Archival Authority Records as Linked Data
thanks to Wikidata, schema.org and the Records in Context Ontology.
ICARUS (International Centre for Archival Research) Convention
\enquote{Archives and Archival Research in the Digital Environment},
Belgrad, Serbien, 2019-09-23 bis 2019-09-25.
\url{https://doi.org/10.5281/zenodo.3465304}

Fabiana Guernaccini, Silvia Mazzini, Giovanni Bruno. (2019). LOD
publication in the archival domain: methods and practices. In \emph{Open
Data and Ontologies for Cultural Heritage. Proceedings of the First
International Workshop on Open Data and Ontologies for Cultural Heritage
co-located with the 31st International Conference on Advanced
Information Systems Engineering (CAiSE 2019)}, hrsg. von Antonella
Poggi: 15--26.
\href{http://ceur-ws.org/Vol-2375/}{http://ceur-ws.org/Vol-2375}

Joachim Neubert. (2017). Wikidata as a linking hub for knowledge
organization systems? Integrating an authority mapping into Wikidata and
learning lessons for KOS mappings. In \emph{Proceedings of the 17th
European Networked Knowledge Organization Systems Workshop co-located
with the 21st International Conference on Theory and Practice of Digital
Libraries 2017 (TPDL 2017)}, hrsg. von Philipp Mayr, Douglas Tudhope,
Koraljka Golub, Christian Wartena and Ernesto William De Luca: 14--25.
\url{http://ceur-ws.org/Vol-1937}

Theo van Veen. (2019). Wikidata: from \enquote{an} Identifier to
\enquote{the} Identifier. In \emph{Information Technology and
Libraries}, 38(2), 72--81.
\url{https://doi.org/10.6017/ital.v38i2.10886}

Jakob Voß. (2017). Normdaten-Mappings in Wikidata. Subject Indexing \&
Information Technology Workshop (SIIT), Göttingen, Germany, 2017-05-11.
\url{https://doi.org/10.5281/zenodo.574452}

\hypertarget{websites}{%
\section{Websites}\label{websites}}

\url{https://www.w3.org/TR/n-triples/}

\url{https://www.wikidata.org/}

\url{https://lov.linkeddata.es/}

\url{https://wiki.dbpedia.org/}

\url{https://d-nb.info/standards/elementset/gnd}

\url{http://xmlns.com/foaf/spec/}

\url{https://dublincore.org/specifications/dublin-core/dcmi-terms/}

\url{http://purl.org/ica/ric}

\url{https://schema.org/}

%autor
\begin{center}\rule{0.5\linewidth}{0.5pt}\end{center}

\textbf{Francesco Gelati} (1987) erwarb die Master-Abschlüsse Linguistik
an der Universität Ca' Foscari Venedig und Geschichte an der Universität
Straßburg. Auch besuchte er die Schule für Archivwissenschaft des
Staatsarchivs zu Venedig. Von 2017 bis 2019 arbeitete er beim Belgischen
Staatsarchiv als Datenimportmanager und ist seit 2019 Archivar am
Institut für Zeitgeschichte München - Berlin. Er interessiert sich für
digitale archivalische Standards, Linked Data und
Forschungsdatenmanagement. ORCID ID:
\url{https://orcid.org/0000-0002-6066-1308}

\end{document}
