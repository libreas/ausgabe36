\documentclass[a4paper,
fontsize=11pt,
%headings=small,
oneside,
numbers=noperiodatend,
parskip=half-,
bibliography=totoc,
final
]{scrartcl}

\usepackage[babel]{csquotes}
\usepackage{synttree}
\usepackage{graphicx}
\setkeys{Gin}{width=.4\textwidth} %default pics size

\graphicspath{{./plots/}}
\usepackage[ngerman]{babel}
\usepackage[T1]{fontenc}
%\usepackage{amsmath}
\usepackage[utf8x]{inputenc}
\usepackage [hyphens]{url}
\usepackage{booktabs} 
\usepackage[left=2.4cm,right=2.4cm,top=2.3cm,bottom=2cm,includeheadfoot]{geometry}
\usepackage{eurosym}
\usepackage{multirow}
\usepackage[ngerman]{varioref}
\setcapindent{1em}
\renewcommand{\labelitemi}{--}
\usepackage{paralist}
\usepackage{pdfpages}
\usepackage{lscape}
\usepackage{float}
\usepackage{acronym}
\usepackage{eurosym}
\usepackage{longtable,lscape}
\usepackage{mathpazo}
\usepackage[normalem]{ulem} %emphasize weiterhin kursiv
\usepackage[flushmargin,ragged]{footmisc} % left align footnote
\usepackage{ccicons} 
\setcapindent{0pt} % no indentation in captions

%%%% fancy LIBREAS URL color 
\usepackage{xcolor}
\definecolor{libreas}{RGB}{112,0,0}

\usepackage{listings}

\urlstyle{same}  % don't use monospace font for urls

\usepackage[fleqn]{amsmath}

%adjust fontsize for part

\usepackage{sectsty}
\partfont{\large}

%Das BibTeX-Zeichen mit \BibTeX setzen:
\def\symbol#1{\char #1\relax}
\def\bsl{{\tt\symbol{'134}}}
\def\BibTeX{{\rm B\kern-.05em{\sc i\kern-.025em b}\kern-.08em
    T\kern-.1667em\lower.7ex\hbox{E}\kern-.125emX}}

\usepackage{fancyhdr}
\fancyhf{}
\pagestyle{fancyplain}
\fancyhead[R]{\thepage}

% make sure bookmarks are created eventough sections are not numbered!
% uncommend if sections are numbered (bookmarks created by default)
\makeatletter
\renewcommand\@seccntformat[1]{}
\makeatother

% typo setup
\clubpenalty = 10000
\widowpenalty = 10000
\displaywidowpenalty = 10000

\usepackage{hyperxmp}
\usepackage[colorlinks, linkcolor=black,citecolor=black, urlcolor=libreas,
breaklinks= true,bookmarks=true,bookmarksopen=true]{hyperref}
\usepackage{breakurl}


%meta

%meta

\fancyhead[L]{R. Gey, K. Leinweber, A. Struck, R. Strötgen, C. Pietsch \\ %author
LIBREAS. Library Ideas, 36 (2019). % journal, issue, volume.
\href{http://nbn-resolving.de/}
{}} % urn 
% recommended use
%\href{http://nbn-resolving.de/}{\color{black}{urn:nbn:de...}}
\fancyhead[R]{\thepage} %page number
\fancyfoot[L] {\ccLogo \ccAttribution\ \href{https://creativecommons.org/licenses/by/4.0/}{\color{black}Creative Commons BY 4.0}}  %licence
\fancyfoot[R] {ISSN: 1860-7950}

\title{\LARGE{Workshop @ deRSE19: Libraries for Research Software \& Engineers}} % title
\author{Ronny Gey, Katrin Leinweber, Alexander Struck, \\Robert Strötgen, Christian Pietsch} % author

\setcounter{page}{1}

\hypersetup{%
      pdftitle={Workshop @ deRSE19: Libraries for Research Software \& Engineers},
      pdfauthor={Ronny Gey, Katrin Leinweber, Alexander Struck, Robert Strötgen, Christian Pietsch},
      pdfcopyright={CC BY 4.0 International},
      pdfsubject={LIBREAS. Library Ideas, 36 (2019).},
      pdfkeywords={research software, RSE Conference, Research Software Engineer, software engineering, software discovery, research data},
      pdflicenseurl={https://creativecommons.org/licenses/by/4.0/},
      pdfcontacturl={http://libreas.eu},
      baseurl={http://libreas.eu},
      pdflang={en},
      pdfmetalang={en}
     }



\date{}
\begin{document}

\maketitle
\thispagestyle{fancyplain} 

%abstracts

%body
The workshop was hosted by five people from library and infrastructure
environments. As information managers we discussed what libraries are
already doing and should do in the future to further explore software
engineering and engineers. The topic attracted participants with diverse
backgrounds in research software engineering and library and information
science. Our goals were to generate/strengthen awareness of the topic of
Research Software \& Engineers {[}RS(E){]} in libraries, to jointly
identify fields of action for libraries, to collect and collaboratively
develop ideas and materials for use in libraries, and to interconnect
the RSE community with libraries.

\hypertarget{introduction}{%
\section{Introduction}\label{introduction}}

Katrin Leinweber from TIB Hannover opened and moderated the workshop.
After a short round of introductions of all the participants, we started
to present the three subtopics of this workshop in short impulse
lectures, from which the participants then choose one to persue. For the
treatment of the topics we decided to use the Speedboat
procedure\footnote{``Speedboat procedure'':
  \url{https://klaxoon.com/blog/speed-boat-an-agile-method-to-discover}.}
known from agile project management. We worked with a diagram of a
sailing boat and an island. The anchor of the sailboat represents the
challenges of the respective topic. The wind in the sails gives the
company strength and stands for aspects that help us to achieve our
goal. The goal in turn is embodied by the island, which represents the
desired state for the respective topic.

\hypertarget{software-discovery-and-its-tools}{%
\section{Software Discovery and its
tools}\label{software-discovery-and-its-tools}}

Discussions revealed that developers utilize general purpose search
engines to find code snippets, for example in order to remind themselves
how a certain algorithm was implemented. Software for re-use purpose is
also searched for in general purpose search engines and within the
researcher's social network.

BASE\footnote{BASE: \url{https://www.base-search.net/}.} as a search
engine for academic audiences does show software from over 7000 sources
it harvests. User Interface refinements to support this use case were
discussed, such as aggregating multiple versions of the same software
into a single entry, with the latest version displayed primarily. It was
also suggested to implement full text and source code search (in
addition to metadata search) in order to increase its usefulness.
Zenodo.org\footnote{Zenodo -- Type ``Software'':
  \url{https://zenodo.org/search?page=1\&size=20\&q=\&type=software}.}
becomes more popular as a software repository due to its integration
with GitHub\footnote{GitHub ``Making Your Code Citable''
  \url{https://guides.github.com/activities/citable-code/}.} which
allows publication with a DOI and thus proper citation\footnote{Research
  Software Citation: \url{https://cite.research-software.org/}.}
metadata. Software Heritage\footnote{Software Heritage:
  \url{https://www.softwareheritage.org/}.} was shortly discussed as an
archival effort to prevent loss of research software as it may happen on
code collaboration platforms. We finished the discussion with ideas on
how to make software and the relevant repositories more visible. A
registry similar to what re3data\footnote{re3data:
  \url{https://re3data.org/}.} does for research data could be a
promising endeavour.

\hypertarget{open-educational-resources}{%
\section{Open Educational
Resources}\label{open-educational-resources}}

In the second track of the workshop, we set out to collect existing Open
Educational Resources on RSE and to show why libraries can be good
partners of the RSE community in terms of education and training. We
agreed in advance that mastery of the tools is particularly important
for research and that software has an outstanding importance as a tool
today. First, we collected existing offers for training
(Carpentries\footnote{The Carpentries: \url{https://carpentries.org/}.}
-- Software\footnote{Software Carpentry:
  \url{https://software-carpentry.org/}.}, Data\footnote{Data Carpentry:
  \url{https://datacarpentry.org/}.}, Library\footnote{Library
  Carpentry: \url{https://librarycarpentry.org/}.},
ProgrammingHistorian\footnote{The Programming Historian:
  \url{https://programminghistorian.org/}.}, Exercism\footnote{Exercism:
  \url{https://exercism.io/}.}), for documentation (Read the
Docs\footnote{Read the Docs -- Create, host, and browse documentation:
  \url{https://readthedocs.org/}.}, MkDocs\footnote{MkDocs -- Project
  documentation with Markdown: \url{https://www.mkdocs.org/}.}), and
collections of interesting aspects in general (awesome
lists\footnote{For example list on GitHub: ``awesome -- Awesome lists
  about all kinds of interesting topics''
  \url{https://github.com/sindresorhus/awesome}.}). We quickly agreed
that there are already many offers and that we as libraries should
rather concentrate on collecting, curating, improving and above all
disseminating the existing offers. We identified potential problems for
this in institutional support -- although the topic of research software
is currently coming to the fore, the necessary structures at the
institutions are still lacking. The offers collected so far are also
based on voluntary work, which is often carried out in leisure time in
addition to regular work. Further, the sheer quantity of available
offers is challenging.

Just as good programmers try to solve a problem by writing as little
code as necessary\footnote{David Strauss: ``All Code is Debt'', February
  12, 2014, \url{https://pantheon.io/blog/all-code-debt}.}, re-using
well-established, shared code libraries, knowledge workers should take
on the challenge of contributing to existing resources, rather than
creating their own. Although the latter is a common criterion in
evaluations, promotions et cetera, it exacerbates the problem of
curating or even just reviewing the sheer quantity of available
material.

\hypertarget{rse-and-the-management-of-research-data}{%
\section{RSE and the Management of Research
Data}\label{rse-and-the-management-of-research-data}}

The third group discussed the close connection of research software and
research data. In many cases publication and archival of research data
without the used research software is useless. This causes several
challenges for example for software of measure devices and simulation.
Libraries have to offer new services in close cooperation with
researchers and research software developers. Libraries have useful
knowledge on aspects like metadata, protocols, vocabularies/ontologies,
persistent identifiers. This knowledge may lead to new services about
reproducibility, discovery, free or restricted access to research data
and software.

Research is changing and becoming more complex. Diversity of
researchers' needs and use cases require flexible cooperation between
librarians, researchers and software developers. Small and concrete
pilot projects should help to develop new ways of cooperation.

\hypertarget{future-work}{%
\section{Future Work}\label{future-work}}

The organizers consider pursuing the topic in the future. Making
software repositories more accessible may be one activity. In our
experience, libraries will have the most impact by connecting scientists
and users to existing tools, curation forums (GitHub topics\footnote{GitHub
  topics: \url{https://github.com/topics/}.}) and initiatives like
Carpentries\footnote{The Carpentries: \url{https://carpentries.org/}.}
or Open Source Guide\footnote{Open Source Guides:
  \url{https://opensource.guide/}.}. Stay tuned.

\hypertarget{derse-conference-review}{%
\section*{deRSE Conference Review}\label{derse-conference-review}}

After three successful UK conferences for Research Software Engineering
the first RSE conference in Germany\footnote{deRSE19 Conference:
  \url{https://www.de-rse.org/en/conf2019/}.} took place at the Albert
Einstein Science Park\footnote{Albert Einstein Science Park:
  \url{https://en.wikipedia.org/wiki/Albert_Einstein_Science_Park}.} in
Potsdam. The organizers created a wonderful atmosphere and a
sophisticated and balanced schedule. Many talks were recorded and have
been published in a dedicated video repository\footnote{deRSE19 video
  recordings: \url{https://av.tib.eu/series/644}.}. There has been
praise\footnote{Helmholtz Open Science Newsletter vom 24.07.2019
  https://os.helmholtz.de/bewusstsein-schaerfen/newsletter/archiv/newsletter-75-vom-24072019/\#c19002}
and some criticism due to some sponsors involved. There are more
national conferences of this kind scheduled\footnote{For example in the
  Netherlands, see Call for contributions -- NL-RSE19:
  \url{https://nl-rse.org/2019/07/09/NL-RSE-2019.html}.} but rumor has
it that an international conference is in the making.

%autor
\begin{center}\rule{0.5\linewidth}{0.5pt}\end{center}

\textbf{Ronny Gey} studied business informatics at the TU-Chemnitz and
then worked as a research assistant at the University of Leipzig and the
FSU Jena. Since 2018 he has been working as a trainee at the University
Library Leipzig and is studying library and information science at the
IBI at the HU Berlin. \url{https://orcid.org/0000-0003-1028-1670}

\textbf{Katrin Leinweber} studied life sciences, ecology and geology,
then shifted towards process automation, data analysis and software
development. She supports and trains researchers and librarians in these
topics since 2017. \url{https://orcid.org/0000-0001-5135-5758}

\textbf{Alexander Struck} has an academic background in LIS \& CS. He
worked for the content industry and does research on citation networks,
research evaluation and research software discovery. Alexander is CIO of
the Cluster of Excellence -- Matters of Activity.
\url{https://orcid.org/0000-0002-1173-9228}

\textbf{Robert Strötgen} is a historian and information scientist and
has developed scientific software at various institutions such as GESIS
and the Georg Eckert Institute. Since 2016 he has been head of the IT
and research support services department at the University Library of
the TU Braunschweig and is now deputy director.
\url{https://orcid.org/0000-0003-3320-5187}

\textbf{Christian Pietsch} is a computational linguist in Bielefeld
University Library's library technology and knowledge management
department. Recent DFG projects he contributed to include CONQUAIRE
(Continuous quality control for research data to ensure reproducibility)
and ORCID DE. Occasionally, he is involved in running BASE (Bielefeld
Academic Search Engine). \url{https://orcid.org/0000-0001-8778-1273}

\end{document}
