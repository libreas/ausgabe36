\documentclass[a4paper,
fontsize=11pt,
%headings=small,
oneside,
numbers=noperiodatend,
parskip=half-,
bibliography=totoc,
final
]{scrartcl}

\usepackage[babel]{csquotes}
\usepackage{synttree}
\usepackage{graphicx}
\setkeys{Gin}{width=.4\textwidth} %default pics size

\graphicspath{{./plots/}}
\usepackage[ngerman]{babel}
\usepackage[T1]{fontenc}
%\usepackage{amsmath}
\usepackage[utf8x]{inputenc}
\usepackage [hyphens]{url}
\usepackage{booktabs} 
\usepackage[left=2.4cm,right=2.4cm,top=2.3cm,bottom=2cm,includeheadfoot]{geometry}
\usepackage{eurosym}
\usepackage{multirow}
\usepackage[ngerman]{varioref}
\setcapindent{1em}
\renewcommand{\labelitemi}{--}
\usepackage{paralist}
\usepackage{pdfpages}
\usepackage{lscape}
\usepackage{float}
\usepackage{acronym}
\usepackage{eurosym}
\usepackage{longtable,lscape}
\usepackage{mathpazo}
\usepackage[normalem]{ulem} %emphasize weiterhin kursiv
\usepackage[flushmargin,ragged]{footmisc} % left align footnote
\usepackage{ccicons} 
\setcapindent{0pt} % no indentation in captions

%%%% fancy LIBREAS URL color 
\usepackage{xcolor}
\definecolor{libreas}{RGB}{112,0,0}

\usepackage{listings}

\urlstyle{same}  % don't use monospace font for urls

\usepackage[fleqn]{amsmath}

%adjust fontsize for part

\usepackage{sectsty}
\partfont{\large}

%Das BibTeX-Zeichen mit \BibTeX setzen:
\def\symbol#1{\char #1\relax}
\def\bsl{{\tt\symbol{'134}}}
\def\BibTeX{{\rm B\kern-.05em{\sc i\kern-.025em b}\kern-.08em
    T\kern-.1667em\lower.7ex\hbox{E}\kern-.125emX}}

\usepackage{fancyhdr}
\fancyhf{}
\pagestyle{fancyplain}
\fancyhead[R]{\thepage}

% make sure bookmarks are created eventough sections are not numbered!
% uncommend if sections are numbered (bookmarks created by default)
\makeatletter
\renewcommand\@seccntformat[1]{}
\makeatother

% typo setup
\clubpenalty = 10000
\widowpenalty = 10000
\displaywidowpenalty = 10000

\usepackage{hyperxmp}
\usepackage[colorlinks, linkcolor=black,citecolor=black, urlcolor=libreas,
breaklinks= true,bookmarks=true,bookmarksopen=true]{hyperref}
\usepackage{breakurl}

%meta
%meta

\fancyhead[L]{R. Strötgen, D. Beucke, Redaktion LIBREAS \\ %author
LIBREAS. Library Ideas, 36 (2019). % journal, issue, volume.
\href{http://nbn-resolving.de/}
{}} % urn 
% recommended use
%\href{http://nbn-resolving.de/}{\color{black}{urn:nbn:de...}}
\fancyhead[R]{\thepage} %page number
\fancyfoot[L] {\ccLogo \ccAttribution\ \href{https://creativecommons.org/licenses/by/4.0/}{\color{black}Creative Commons BY 4.0}}  %licence
\fancyfoot[R] {ISSN: 1860-7950}

\title{\LARGE{Fünf Fragen zur Nachhaltigkeit von Forschungsinfrastrukturen}} % title
\author{Robert Strötgen, Daniel Beucke, Redaktion LIBREAS} % author

\setcounter{page}{1}

\hypersetup{%
      pdftitle={Fünf Fragen zur Nachhaltigkeit von Forschungsinfrastrukturen},
      pdfauthor={Robert Strötgen, Daniel Beucke, Redaktion LIBREAS},
      pdfcopyright={CC BY 4.0 International},
      pdfsubject={LIBREAS. Library Ideas, 36 (2019).},
      pdfkeywords={Nachhaltigkeit, Forschungsinfrastruktur, Kitodo, SUB Göttingen},
      pdflicenseurl={https://creativecommons.org/licenses/by/4.0/},
      pdfcontacturl={http://libreas.eu},
      baseurl={http://libreas.eu},
      pdflang={de},
      pdfmetalang={de}
     }



\date{}
\begin{document}

\maketitle
\thispagestyle{fancyplain} 

%abstracts

%body
Nicht jede*r, der/die etwas zu berichten hat, findet die Zeit, einen
längeren Beitrag zu erstellen. Das Dilemma kennen auch wir in der
LIBREAS-Redaktion, Wunsch und Wirklichkeit finden nicht immer zusammen.
Ergänzend zum Aufruf, vollständige Beiträge einzureichen, haben wir
(ähnlich wie in Ausgabe \#33 Ortstermin,
\url{https://libreas.eu/ausgabe33/5F5B/intro/}) wieder ein Kurzformat
angeboten. Gewünscht haben wir uns von möglichst vielen Betreiber*innen,
Entwickler*innen und Mitarbeiter*innen von Forschungsinfrastrukturen
Antworten auf die folgenden fünf Fragen.

\begin{itemize}
\item
  An welcher Einrichtung sind Sie tätig und welche Art von
  Forschungsinfrastruktur wird an ihr betrieben bzw. genutzt?
  (Mehrfachnennungen möglich)
\item
  Wie ist das Finanzierungsmodell für diese Forschungsinfrastruktur(en)
  gestaltet?
\item
  Welche Rolle spielt Kollaboration mit anderen Einrichtungen für
  Betrieb und Weiterentwicklung der Forschungsinfrastruktur(en)?
\item
  Welche Rolle spielen freie Lizenzen und andere Kriterien offener
  Wissenschaft in diesem Zusammenhang?
\item
  Ist die Forschungsinfrastruktur aus Ihrer Sicht nachhaltig? Warum
  (nicht)? Falls nicht: Was fehlt, um den nachhaltigen Betrieb
  abzusichern?
\end{itemize}

Die eingereichten Antworten präsentieren wir in diesem Artikel.

\hypertarget{kitodo.-key-to-digital-objects-e.-v.}{%
\section{Kitodo. Key to digital objects e.\,V.}\label{kitodo.-key-to-digital-objects-e.-v.}}

Eingereicht von Robert Strötgen (Vorsitzender des Vorstands von
\enquote{Kitodo. Key to digital objects} e.\,V.)

\textbf{1) An welcher Einrichtung sind Sie tätig und welche Art von
Forschungsinfrastruktur wird an ihr betrieben bzw. genutzt?}

Ich arbeite als stellvertretender Bibliotheksdirektor an
Universitätsbibliothek der TU Braunschweig und leite dort die Abteilung
IT und forschungsnahe Services. Seit knapp zwei Jahren bin ich
Vorsitzender des Vorstands des Vereins \enquote{Kitodo. Key to digital
objects} e.\,V. Dieser Verein hat sich 2012 gegründet, um die
Digitalisierung an Kulturgedächtniseinrichtungen zu fördern und pflegt
dafür unter anderem die Software Kitodo (\url{https://www.kitodo.org/}).
Mit Kitodo.Production existiert ein Workflowsystem, das den
Digitalisierungsprozess (vom Scanner über die Qualitätssicherung und
Metadatenbearbeitung bis zur Veröffentlichung und Archivierung und
gegebenenfalls mit weiteren Prozessschritten) effizient steuert.
Kitodo.Presentation ist eine Präsentationssoftware, die Digitalisate
benutzerfreundlich zugänglich macht. Der Verein hat sich der Offenheit
der Software und des Entwicklungsprozesses verschrieben, um den
Bibliotheken und Archiven eine transparente Mitwirkungsmöglichkeit zu
sichern und dabei eine faire und partnerschaftliche Zusammenarbeit mit
den Dienstleistern zu unterstützen. Die Governancestruktur des Vereins
sorgt für eine Nachhaltigkeit der Open-Source-Software.

\textbf{2) Wie ist das Finanzierungsmodell für diese
Forschungsinfrastruktur(en) gestaltet?}

Der Verein finanziert sich aus Mitgliedsbeiträgen. Mit diesen Mitteln
kann das Releasemanagement der Software bezahlt werden, das essenziell
für das Ziel der Offenheit ist. Darüber hinaus können aus den
Vereinsmitteln Öffentlichkeitsarbeit zur Förderung der Digitalisierung
sowie Weiterbildungsveranstaltungen finanziert werden. Die Entwicklung
der Software wird von einzelnen Mitgliedern und den Dienstleistern
getragen. Erhebliche Fortschritte hat die Software durch ein
dreijähriges DFG-Projekt zur Weiterentwicklung von Kitodo.Production
machen können. Für die weitere Zukunftssicherung bei der Entwicklung
werden im Verein neue Modelle wie zum Beispiel ein Entwicklungsfonds
diskutiert.

\textbf{3) Welche Rolle spielt Kollaboration mit anderen Einrichtungen
für Betrieb und Weiterentwicklung der Forschungsinfrastruktur(en)?}

Der Verein ist aus der Zusammenarbeit entstanden und versteht sich als
institutioneller Rahmen, um diese Zusammenarbeit langfristig zu sichern
und für faire Regeln der Kooperation zu sorgen. Der Verein selbst
entwickelt sich nur durch das Engagement und die Beiträge der Mitglieder
weiter.

\textbf{4) Welche Rolle spielen freie Lizenzen und andere Kriterien
offener Wissenschaft in diesem Zusammenhang?}

Für Kitodo e.\,V. ist eine frei Lizenzierung selbstverständlich. Aber
eine freie Lizenz reicht nach unserer Überzeugung nicht aus. Der ganze
Entwicklungsprozess ist auf Offenheit und Transparenz angelegt. Das
Releasemanagement, das allen Partnern gleichberechtigten Zugang zum
Quelltext und bei der Steuerung von Entwicklungsvorhaben gewährt, ist
dabei ein wichtiges Werkzeug, ein weiteres die im Verein entwickelten
und fortgeschriebenen Coding Guidelines.

\textbf{5) Ist die Forschungsinfrastruktur aus Ihrer Sicht nachhaltig?
Warum (nicht)? Falls nicht: Was fehlt, um den nachhaltigen Betrieb
abzusichern?}

Die Sicherung von Nachhaltigkeit war einer der zentralen Beweggründe zur
Vereinsgründung. Nachhaltigkeit kann aber nur gelebt werden, wenn der
Verein vom Interesse, dem Engagement und den Beiträgen der Mitglieder
getragen wird. Kitodo e.\,V. ist ein sehr vitaler und bislang stetig
wachsender Verein. Trotzdem muss der Verein permanent unter Beweis
stellen, dass seine Arbeit relevant und nützlich für die Mitglieder und
die Community ist, um das Engagement zu erhalten.

\hypertarget{nachhaltigkeit-von-forschungsinfrastrukturen-an-der-sub-guxf6ttingen-drei-beispiele}{%
\section{Nachhaltigkeit von Forschungsinfrastrukturen an der SUB Göttingen -- drei Beispiele}\label{nachhaltigkeit-von-forschungsinfrastrukturen-an-der-sub-guxf6ttingen-drei-beispiele}}

Eingereicht von Daniel Beucke (Mitarbeiter im Bereich Digitale
Bibliothek an der SUB Göttingen)

\textbf{1) An welcher Einrichtung sind Sie tätig und welche Art von
Forschungsinfrastruktur wird an ihr betrieben bzw. genutzt?}

Ich bin Mitarbeiter an der Niedersächsischen Staats- und
Universitätsbibliothek Göttingen (SUB Göttingen). Dort bin ich in der
Gruppe Elektronisches Publizieren und in der Gruppe Software- und
Serviceentwicklung tätig. In diesem Rahmen bin ich für die Konzeption
und den Aufbau von unterschiedlichen Services und Infrastrukturen
mitverantwortlich. Im folgenden möchte ich drei Beispiele nennen, bei
denen ich mitgewirkt habe und die einen guten Überblick über die
unterschiedlichen Forschungsinfrastrukturen geben:

\begin{itemize}
\item
  Die Informationsplattform Open Access (IPOA,
  \url{https://open-access.net/})\\
  Diese Plattform ist im Rahmen eines DFG-geförderten Projekts mit
  mehreren Projektpartnern entstanden. Die Plattform ist ein zentraler
  Ort für Informationen rund um das Thema Open Access. Die SUB Göttingen
  war unter anderem für den Aufbau der technischen Infrastruktur
  zuständig.
\item
  Die Kooperation mit der Akademie der Wissenschaften zu Göttingen (AdW
  Göttingen)\\
  Im Rahmen einer Kooperation mit der AdW Göttingen entstehen seit 2011
  verschiedenen Forschungsinfrastrukturen. Gestartet wurde mit einer
  zentralen Website für die AdW Göttingen (\url{https://adw-goe.de}) und
  einem Dokumentenserver für Open-Access-Publikationen
  (\url{https://rep.adw-goe.de}) der AdW Göttingen. Hinzu kamen mehrere
  Services für die zahlreichen Vorhaben der Akademie, wie eine
  Online-Version des Frühneuhochdeutschen Wörterbuches
  (\url{https://fwb-online.de}), Briefportale für Leibniz
  (\url{http://leibniz-briefportal.adw-goe.de/start}) und Gauß
  (\url{https://gauss.adw-goe.de}) oder ein Verzeichnis der Höfe und
  Residenzen im spätmittelalterlichen Reich
  (\url{https://adw-goe.de/digitale-bibliothek/hoefe-und-residenzen-im-spaetmittelalterlichen-reich}).
\item
  Das Publikationsdatenmanagement GRO.publications
  (\url{https://publications.goettingen-research-online.de/})\\
  Die SUB Göttingen stellt den Forschenden an der Universität Göttingen
  ein neues Publikationsdatenmanagement zur Verfügung, das die
  Erstellung von Publikationsnachweisen wesentlich erleichtert
  beziehungsweise erstmals ermöglicht. Dies erfolgt im Rahmen des neuen
  Services Göttingen Research Online (GRO), der weitere Services wie ein
  Forschungs\-daten-Repositorium oder die Erstellung von
  Datenmanagementplänen anbietet.
\end{itemize}

\textbf{2) Wie ist das Finanzierungsmodell für diese
Forschungsinfrastruktur(en) gestaltet?}

Die zuvor genannten Services bilden einen guten Querschnitt bei der
nachhaltigen Finanzierung von eben solchen Infrastrukturen, die an der
SUB konzipiert, entwickelt und betrieben werden.

Die \textbf{IPOA} wurde in zwei Förderphasen mit Finanzierung von der
DFG entwickelt. Nach Ablauf der Projektförderung verpflichteten sich die
Projektpartner die Plattform weiterhin anzubieten und aktuell zu halten.
Die SUB Göttingen übernimmt dabei den technischen Betrieb, die anderen
Partner zeichnen sich für die Inhalte verantwortlich. Dafür wurde ein
Memorandum of Understanding (MoU) unterzeichnet, das nach Ablauf der
Frist wieder verlängert werden muss.

Die Angebote, die im Rahmen der \textbf{Kooperation mit der AdW
Göttingen} entstanden sind, werden durch eine Kooperation finanziert. Im
Rahmen dieser Vereinbarung, die in der Regel alle zwei Jahre neu
vereinbart wird, werden gemeinsam die anfallenden Arbeiten festgelegt.
Der Schwerpunkt liegt auf der Weiterentwicklung von
Forschungsinfrastruktur für die Vorhaben der AdW Göttingen.

Das \textbf{Publikationsdatenmangement GRO.publications} ist seit Sommer
2019 ein neuer Service für die Universität Göttingen, der von der SUB
Göttingen angeboten wird. Die Konzeptphase und ein Teil der Entwicklung
wurde im Rahmen eines Projekts innerhalb der Universität finanziert. Die
weitere Entwicklung und der Betrieb des Services wird von der SUB
Göttingen aus Eigenmitteln finanziert.

\textbf{3) Welche Rolle spielt Kollaboration mit anderen Einrichtungen
für Betrieb und Weiterentwicklung der Forschungsinfrastruktur(en)?}

Die Zusammenarbeit mit anderen Einrichtungen am Göttingen Campus und
auch darüber hinaus sind für uns sehr wichtig. So arbeiten wir bei dem
Betrieb von Forschungsinfrastrukturen eng mit der Gesellschaft für
wissenschaftliche Datenverarbeitung mbH Göttingen (GWDG) zusammen, die
größtenteils die technische Grundlage -- als Rechenzentrum der
Universität Göttingen -- bereitstellt. Aus dieser Zusammenarbeit
entstand auch die Göttingen eResearch Alliance (ein Zusammenschluss der
GWDG und der SUB Göttingen,
\url{https://www.eresearch.uni-goettingen.de}), die die Services rund um
Göttingen Research Online anbietet und vor allem für das
Forschungsdatenmanagment am Standort verantwortlich ist.

Aber auch die Zusammenarbeit mit nationalen und internationalen Partnern
ist in der SUB Göttingen von großer Bedeutung. So ist die SUB Göttingen
Projektpartner in zahlreichen Projekten zu Forschungsinfrastrukturen
(zum Beispiel OpenAIRE, DARIAH-DE). Die Zusammenarbeit mit anderen
Einrichtungen in den zahlreichen Projekten
(\url{https://www.sub.uni-goettingen.de/projekte-forschung/projekte-a-z/})
ermöglicht uns einen breiteren Kreis an Nutzenden zu erreichen.

Für den sinnvollen Betrieb einer Forschungsinfrastruktur sollen die
organisatorische, technische und fachliche Zuständigkeit spätestens am
Ende der Projektphase definiert sein. Diese Aufgaben können zwischen
verschiedenen Einrichtungen aufgeteilt werden, was de facto zu einem
kollaborativen Betrieb führt. Dieses Modell hat den Vorteil, dass die
Betriebskosten nicht von einer einzelnen Einrichtung getragen werden
müssen und die Einrichtungen sich auf ihre Kernaufgaben konzentrieren
können (siehe Beispiel IPOA). Die Finanzierung der Betriebskosten
erfolgt meist mit der Eigenleistung der Trägereinrichtungen, da es nur
wenige erfolgreiche Geschäftsmodelle für den Betrieb der
Forschungsinfrastrukturen gibt, vor allem wenn sie im Rahmen von
Drittmittelprojekten entstanden sind. Die Weiterentwicklung erfolgt in
der Regel auch kollaborativ -- aus ähnlichen Gründen. Leider gibt es
hier auch zu wenig Geschäftsmodelle, die den Einrichtungen helfen, die
erforderlichen Kapazitäten bereit zu halten.

\pagebreak
\textbf{4) Welche Rolle spielen freie Lizenzen und andere Kriterien
offener Wissenschaft in diesem Zusammenhang?}

Die Nutzung von freien Lizenzen ist im Rahmen der Entwicklung in der SUB
Göttingen von großer Bedeutung und auch in einer Policy
verankert.\footnote{Siehe auch die \enquote{Digital Policies: Grundsätze
  für die digitalen Angebote der Niedersächsischen Staats- und
  Universitätsbibliothek Göttingen} und dort zur IT-Architektur;
  \url{https://www.sub.uni-goettingen.de/wir-ueber-uns/portrait/digital-policies-grundsaetze-fuer-die-digitalen-angebote-der-sub-goettingen/it-architektur/}.}
Bei den verschiedenen Services wird darauf geachtet, dass
Neuentwicklungen oder Anpassungen an bereits existierender Software frei
über ein Git-Repositorium angeboten werden. Dabei nutzen wir in der
Regel GitHub (\url{https://github.com/subugoe}) für Entwicklungen, die
von allen nachgenutzt werden können. Für interne Projekte steht uns
ebenfalls eine GitLab-Instanz zur Verfügung, die durch die GWDG
angeboten wird.

\textbf{5) Ist die Forschungsinfrastruktur aus Ihrer Sicht nachhaltig?
Warum (nicht)? Falls nicht: Was fehlt, um den nachhaltigen Betrieb
abzusichern?}

Grundsätzlich würde ich sagen, dass die Infrastrukturen nachhaltig
angeboten werden können. Im Einzelnen hängt dies von den
Grundvoraussetzungen ab.

Schwieriger sind sicher Services, die aus geförderten Projekten
entstehen wie beispielsweise die genannte Informationsplattform Open
Access. Allerdings zeigt dieses Beispiel auch sehr gut, dass der Betrieb
auf verteilen Schultern angeboten werden kann. Das MoU hilft hierbei
sicher.

Allgemein entsteht durch die Digitalisierung das Problem, dass der
Bedarf an digitalen Services schneller steigt als die digitalen und
finanziellen Kapazitäten der betreibenden Einrichtungen. Auch der
Fachkräftemangel macht sich stark bemerkbar, zusammen mit den
verwaltungstechnischen Schwierigkeiten der Personalbindung in
öffentlichen Einrichtungen. Für den nachhaltigen Betrieb brauchen wir
ein besser skalierendes Modell (Betrieb von mehr Services mit
gleichbleibenden Kapazitäten), Aufgabenkritik um wenig genutzte Services
abzuschalten, attraktive Arbeitsbedingungen für Fachkräfte und eine
agilere Verwaltung. Dass die Weiterentwicklung einer
Forschungsinfrastruktur fester Bestandteil des Betriebskonzepts ist,
sollte nicht ignoriert werden.

%autor

\end{document}
