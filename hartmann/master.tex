\documentclass[a4paper,
fontsize=11pt,
%headings=small,
oneside,
numbers=noperiodatend,
parskip=half-,
bibliography=totoc,
final
]{scrartcl}

\usepackage[babel]{csquotes}
\usepackage{synttree}
\usepackage{graphicx}
\setkeys{Gin}{width=.4\textwidth} %default pics size

\graphicspath{{./plots/}}
\usepackage[ngerman]{babel}
\usepackage[T1]{fontenc}
%\usepackage{amsmath}
\usepackage[utf8x]{inputenc}
\usepackage [hyphens]{url}
\usepackage{booktabs} 
\usepackage[left=2.4cm,right=2.4cm,top=2.3cm,bottom=2cm,includeheadfoot]{geometry}
\usepackage{eurosym}
\usepackage{multirow}
\usepackage[ngerman]{varioref}
\setcapindent{1em}
\renewcommand{\labelitemi}{--}
\usepackage{paralist}
\usepackage{pdfpages}
\usepackage{lscape}
\usepackage{float}
\usepackage{acronym}
\usepackage{eurosym}
\usepackage{longtable,lscape}
\usepackage{mathpazo}
\usepackage[normalem]{ulem} %emphasize weiterhin kursiv
\usepackage[flushmargin,ragged]{footmisc} % left align footnote
\usepackage{ccicons} 
\setcapindent{0pt} % no indentation in captions

%%%% fancy LIBREAS URL color 
\usepackage{xcolor}
\definecolor{libreas}{RGB}{112,0,0}

\usepackage{listings}

\urlstyle{same}  % don't use monospace font for urls

\usepackage[fleqn]{amsmath}

%adjust fontsize for part

\usepackage{sectsty}
\partfont{\large}

%Das BibTeX-Zeichen mit \BibTeX setzen:
\def\symbol#1{\char #1\relax}
\def\bsl{{\tt\symbol{'134}}}
\def\BibTeX{{\rm B\kern-.05em{\sc i\kern-.025em b}\kern-.08em
    T\kern-.1667em\lower.7ex\hbox{E}\kern-.125emX}}

\usepackage{fancyhdr}
\fancyhf{}
\pagestyle{fancyplain}
\fancyhead[R]{\thepage}

% make sure bookmarks are created eventough sections are not numbered!
% uncommend if sections are numbered (bookmarks created by default)
\makeatletter
\renewcommand\@seccntformat[1]{}
\makeatother

% typo setup
\clubpenalty = 10000
\widowpenalty = 10000
\displaywidowpenalty = 10000

\usepackage{hyperxmp}
\usepackage[colorlinks, linkcolor=black,citecolor=black, urlcolor=libreas,
breaklinks= true,bookmarks=true,bookmarksopen=true]{hyperref}
\usepackage{breakurl}

\appto\UrlBreaks{\do\a\do\b\do\c\do\d\do\e\do\f\do\g\do\h\do\i\do\j
\do\k\do\l\do\m\do\n\do\o\do\p\do\q\do\r\do\s\do\t\do\u\do\v\do\w
\do\x\do\y\do\z}

%meta
%meta

\fancyhead[L]{N.K. Hartmann \\ %author
LIBREAS. Library Ideas, 36 (2019). % journal, issue, volume.
\href{http://nbn-resolving.de/}
{}} % urn 
% recommended use
%\href{http://nbn-resolving.de/}{\color{black}{urn:nbn:de...}}
\fancyhead[R]{\thepage} %page number
\fancyfoot[L] {\ccLogo \ccAttribution\ \href{https://creativecommons.org/licenses/by/4.0/}{\color{black}Creative Commons BY 4.0}}  %licence
\fancyfoot[R] {ISSN: 1860-7950}

\title{\LARGE{Personenbezogene Forschungsdaten in unverdächtigen Disziplinen: Das Beispiel der Erd-, Umwelt- und Agrarwissenschaften}} % title
\author{Niklas K. Hartmann} % author

\setcounter{page}{1}

\hypersetup{%
      pdftitle={Personenbezogene Forschungsdaten in unverdächtigen Disziplinen: Das Beispiel der Erd-, Umwelt- und Agrarwissenschaften},
      pdfauthor={Niklas K. Hartmann},
      pdfcopyright={CC BY 4.0 International},
      pdfsubject={LIBREAS. Library Ideas, 36 (2019).},
      pdfkeywords={Forschungsdaten, Datenschutz, Geowissenschaften, Umweltwissenschaften, Agrarwissenschaft},
      pdflicenseurl={https://creativecommons.org/licenses/by/4.0/},
      pdfcontacturl={http://libreas.eu},
      baseurl={http://libreas.eu},
      pdflang={de},
      pdfmetalang={de}
     }



\date{}
\begin{document}

\maketitle
\thispagestyle{fancyplain} 

%abstracts
\begin{abstract}
\noindent
Am Beispiel der Erd- und Umweltwissenschaften (einschließlich der
landschafts- und standortbezogenen Teilgebiete der Agrarwissenschaften)
zeigt dieser Beitrag, dass auch in scheinbar „unverdächtigen``
Disziplinen personenbezogene Forschungsdaten vorkommen. Eine Auswertung
der Literatur zeigt, dass allgemeine Handreichungen zum Datenschutz in
der Forschung kaum Unterstützung bei der Arbeit mit den für diese
Disziplinen besonders relevanten Fällen bieten. Für die in den Erd- und
Umweltwissenschaften besonders relevanten raumbezogenen Daten kommt
hinzu, dass selbst unter Fachjuristinnen Uneinigkeit über die
datenschutzrechtliche Bewertung herrscht. Die Ergebnisse einer
empirischen Vorstudie zeigen eine ganze Reihe verschiedener Arten
personenbezogener Forschungsdaten auf, die in der Forschungspraxis der
Erd- und Umweltwissenschaften eine Rolle spielen. Sie legen außerdem
nahe, dass der Umgang mit personenbezogenen Daten in der
Forschungspraxis der Erd- und Umweltwissenschaften auf Grund der
mangelnden Vertrautheit mit dem Datenschutz nicht immer den rechtlichen
Anforderungen entspricht. Auch Unterstützung durch Fachgesellschaften
und Infrastruktureinrichtungen -- etwa in Form disziplinspezifischer
Handreichungen, qualifizierter Beratung oder institutionalisierten
Möglichkeiten, Daten sicher zu archivieren und gegebenenfalls
zugangsbeschränkt zu publizieren -- bestehen kaum. Aus dieser Situation
ergeben sich Herausforderungen an die Weiterentwicklung der
disziplinären Datenkultur und Dateninfrastruktur, beispielsweise im
Rahmen des Prozesses zum Aufbau einer Nationalen
Forschungsdateninfrastruktur (NFDI). Zu den Möglichkeiten für
Infrastruktureinrichtungen, diese Weiterentwicklung zu unterstützen,
zeigt dieser Beitrag Handlungsoptionen auf.
\end{abstract}

%body
\hypertarget{einleitung}{%
\section{Einleitung}\label{einleitung}}

Der Begriff der Nachhaltigkeit bleibt problematisch. Gerade in seiner
nahezu beliebigen Ausweitung als Synonym für \enquote{gut gemacht, und
zwar mehr als nur kurzfristig} kann das Reden über Nachhaltigkeit
maßgeblich zum Erhalt offensichtlich nicht dauerhaft tragfähiger
Konfigurationen beitragen, wie Blühdorn~(2007) für die umwelt- und
entwicklungspolitische Herkunftsdomäne des Begriffs zeigt. Das ist
insbesondere dann der Fall, wenn die Konzentration auf die Effizienz
einzelner Elemente und Ablä ufe den Blick auf schwierige und
konfliktträchtige Aspekte des Ganzen verstellt: Auch die Steuerzentrale
eines Braunkohlekraftwerks kann mit zertifizierter Green IT ausgestattet
werden. Davon mag die Betreiberin auch einfacher zu überzeugen sein als
von einer Stilllegung. Die CO\textsubscript{2}-Bilanz wird das jedoch
kaum beeinflussen.

Zugang zu Forschungsdaten ist nur dann \enquote{nachhaltig} (im Sinne
von längerfristig sicherzustellen), wenn er nicht ausschließlich
informell gewährt wird. Daten, die nicht an eine
Forschungsdateninfrastruktur übergeben werden, überleben selten die
Emeritierung der Arbeitsgruppenleitung (vergleiche Vines \emph{et al.}
2013). Damit Forschungsdaten an eine solche Infrastruktur übergeben
werden können, müssen die Daten -- neben vielen anderen Kriterien --
entweder juristisch und forschungsethisch unbedenklich sein oder die
Infrastruktur muss in angemessener Art und Weise beschränkten Zugang
gewähren können. Für Forschungsgebiete, in denen regelmäßig
\enquote{problematische} Daten verarbeitet werden, müssen die relevanten
Forschungsdateninfrastrukturen also in Abstimmung mit den Forschenden
die nötigen rechtlichen und ethischen Standards, Kompetenzen sowie die
organisatorischen und technischen Voraussetzungen entwickeln. Geschieht
dies nicht, können die Infrastrukturen -- egal wie \enquote{nachhaltig}
sie im technischen und im ökologischen Sinne betrieben werden -- für das
jeweilige Gebiet relevante Daten nicht \enquote{nachhaltig}
bereitstellen.

Ist von personenbezogenen Forschungsdaten die Rede, geht es meist um die
Sozialwissenschaften, die Humanmedizin oder um andere Wissenschaften,
für die der Mensch im Mittelpunkt des Erkenntnisinteresses steht, wie
Psychologie und Pädagogik. In diesen Fächern ist ein Grundverständnis
für den Datenschutz meist ein Teil der Methodenausbildung. Oft haben
Fachgesellschaften bereits Leitlinien herausgegeben und
disziplinspezifische Forschungsdateninfrastrukturen bieten
spezialisierte Beratung an sowie die Möglichkeit, Daten zu archivieren
und zugangsbeschränkt zur Nachnutzung bereitzustellen. Auch mit
forschungsethischen Erwägungen mit Bezug auf Forschungsdaten über den
rechtlich erforderlichen Datenschutz hinaus gibt es meist eine gewisse
Vertrautheit.

In meiner Beratungstätigkeit im Rahmen meiner Aufgabe als Koordinator
Forschungsdaten an der Universität Potsdam begegnen mir jedoch immer
wieder Fälle, in denen in scheinbar \enquote{unverdächtigen} Disziplinen
Forschungsdaten verarbeitet werden, die Personenbezug haben, deren
Personenbezug unklar ist oder die aus Gründen der Forschungsethik nicht
allgemein verfügbar werden dürfen. Oft fällt dies erst auf, wenn Daten
mit hohem Nachnutzungspotenzial veröffentlicht werden sollen.
Einzelfälle kann ich natürlich immer von dem
Forschungsdaten-Kontaktpunkt an Stellen wie die Datenschutzbeauftragte
oder die Ethikkommission weiterverweisen. Häufen sich jedoch Fälle aus
einzelnen Bereichen, liegt eine Bestandsaufnahme nahe, um festzustellen,
ob Handlungsbedarf besteht und wenn ja, für wen (zum Beispiel Strukturen
des Faches wie Fachgesellschaften oder disziplinspezifische
Infrastruktureinrichtungen oder generische Strukturen wie
Infrastruktureinrichtungen der Hochschulen) und in welchem Rahmen.
Aktivitäten dieser Art können auch zusätzliche Handlungsoptionen
eröffnen, die zu einem früheren Zeitpunkt im
Forschungsprojekt-Managementzyklus beziehungsweise
Forschungsdaten-Lebenszyklus ansetzen als erst zur Veröffentlichung.

In diesem Beitrag erfolgt eine solche Bestandsaufnahme für das Gebiet
der Erd- und Umweltwissenschaften, einschließlich der landschafts- und
standortbezogenen Teilgebiete der Agrarwissenschaften. Aus diesem
Bereich haben mich in der Vergangenheit Anfragen erreicht und meine
Ausbildung als Diplom-Geoökologe gibt mir den nötigen Einblick in die
Eigenheiten der betroffenen Fächer. Im Folgenden stelle ich zunächst die
Forschungsdaten der Erd- und Umweltwissenschaften vor (Abschnitt~2),
diskutiere dann ihren möglichen Personenbezug (Abschnitt~3), präsentiere
die Ergebnisse einer empirischen Vorstudie (Abschnitt~4), leite
schließlich Handlungsempfehlungen insbesondere für
Infrastruktureinrichtungen ab (Abschnitt~5) und schließe mit einem Fazit
samt Ausblick (Abschnitt~6).

\hypertarget{forschungsdaten-in-den-erd--und-umweltwissenschaften}{%
\section{Forschungsdaten in den Erd- und
Umweltwissenschaften}\label{forschungsdaten-in-den-erd--und-umweltwissenschaften}}

Ein Kennzeichen von Forschungsdaten in den Erd- und Umweltwissenschaften
ist ihre Heterogenität. Dabei gehen nicht nur verschiedene
Teildisziplinen mit sehr unterschiedlichen Daten um, vielmehr wird
selbst innerhalb einzelner Forschungsvorhaben oft mit heterogenen Daten
gearbeitet. Die Daten unterscheiden sich nicht nur in ihrem Inhalt
(welche Aspekte welcher Umweltkompartimente sie wie repräsentieren),
sondern auch in ihrer Art und ihrem Entstehungskontext.

In vielen Teildisziplinen und Forschungsfeldern spielen nicht
replizierbare und damit besonders wertvolle Beobachtungsdaten eine
ebenso große Rolle wie experimentelle Daten. Sowohl experimentelle als
auch Beobachtungsdaten können je nach Fragestellung im Labor oder im
Freiland gewonnen werden. Die Daten können als qualitative
Beschreibungen in verschiedenen Graden der Standardisierung ebenso
vorliegen wie als quantitative Zähl- und Messdaten. Wo Land unter
menschlicher Nutzung betroffen ist, werden im Freiland gewonnene Daten
teils mit Einverständnis von Eigentümerinnen oder anderen
Verantwortlichen erhoben (zum Beispiel beim Nehmen von Bodenproben oder
bei der Arbeit über Wildtiere in Schutzgebieten), teils gemeinsam mit
für die Kooperation entlohnten Landmanagerinnen generiert (zum Beispiel
in Feldversuchen), teils aber auch ganz ohne Zutun der Landnutzerinnen
verarbeitet (etwa bei der Nutzung von Fernerkundungsdaten oder Daten aus
Kartierungen und Zählungen). Viele dieser Daten haben einen expliziten
Raumbezug, auch Zeitreihen werden regelmäßig erzeugt. Zur Verarbeitung
der Daten werden oft Geographische Informationssysteme oder
spezialisierte Forschungssoftware genutzt. Das Ergebnis sind oftmals
komplexe Datenprodukte, teils in Form (digitaler) Karten. Die
Entstehungskontexte der Daten reichen von Individualforschung über
Arbeitsgruppen und kleinere Verbundprojekte bis zu großen
Forschungsinfrastrukturen wie etwa Satelliten der Erdbeobachtung oder
Dauerfeldversuchen der Agrar- und Forstwissenschaften beziehungsweise
Dauerbeobachtungsflächen der terrestrischen Umweltforschung. Auf Grund
ihrer Heterogenität können die meisten Daten der Erd- und
Umweltwissenschaften, die nicht aus Forschungsinfrastrukturen stammen,
aus Sicht der Dateninfrastrukturen als \emph{long tail}-Daten bezeichnet
werden (vergleiche Klump~2012; Bertelmann~\emph{et al.}~2014).

Während insbesondere in den Geowissenschaften \emph{sensu strictu} Daten
aus Großprojekten bereits seit langem von Dateninfrastrukturen
vorgehalten und international verfügbar gemacht werden, gibt es für
\emph{long tail}-Daten in vielen Teildisziplinen eine verbreitete Kultur
des informellen Teilens. Standardisierte Datenbanken (wie etwa für
Sequenz- und Strukturdaten in den Biowissenschaften) entwickelten sich
auf Grund der Heterogenität der Daten daher zunächst nicht. In den
letzten etwa zehn Jahren waren die Erd- und Umweltwissenschaften jedoch
eines der Wissenschaftsgebiete, in denen das digitale Publizieren von
Forschungsdaten große Fortschritte gemacht hat~(vergleiche
Diepenbroek~2011, Klump~2012, Bertelmann~\emph{et al.}~2014). Die
Recherchierbarkeit und Zitierbarkeit von Daten aus Großprojekten wurde
ebenso verbessert wie die Publikation von Daten aus Einzelprojekten, die
verstärkt nicht nur von Fördermittelgeberinnen, sondern auch von
Fachzeitschriften gefordert wird~(vergleiche COPDESS~2015; Hanson,
Lehnert und Cutcher-Gershenfeld~2015; McNutt~\emph{et al.}~2016;
Bertelmann~2017). Einen aktuellen Überblick über die relevanten
disziplinspezifischen Forschungsdateninfrastrukturen geben die Extended
Abstracts und Absichtserklärungen der geplanten NFDI-Konsortien,
NFDI4Agri, NFDI4Biodiversity und NFDI4Earth.\footnote{Abrufbar unter
  \url{https://www.dfg.de/foerderung/programme/nfdi/absichtserklaerungen/index.html}.}

Eine große Relevanz für die Erd- und Umweltwissenschaften haben auch
allgemein verfügbare oder auf Anfrage erhältliche Daten und
Datenprodukte von Behörden, insbesondere Karten und Geodaten sowie Daten
der EU-Agrarförderung. Diese wichtigen Datengrundlagen sind innerhalb
der EU mittlerweile oft über die Online-Portale staatlicher
Geodateninfrastrukturen verfügbar~(vergleiche Bernard~\emph{et
al.}~2016).

\hypertarget{personenbezug-erd--und-umweltwissenschaftlicher-forschungsdaten}{%
\section{Personenbezug erd- und umweltwissenschaftlicher
Forschungsdaten}\label{personenbezug-erd--und-umweltwissenschaftlicher-forschungsdaten}}

\enquote{‚Personenbezogene Daten' {[}sind{]} alle Informationen, die
sich auf eine identifizierte oder identifizierbare Person beziehen}
(Art.~4, Abs.~1 Datenschutzgrundverordnung der Europäischen Union,
DS-GVO). Um Informationen zu sein, die sich auf eine Person beziehen,
muss nicht nur eine Person identifizierbar sein, sondern die Daten
müssen auch tatsächlich Angaben über diese Person enthalten (Gusy und
Eichendorfer~2018, Rn.~47). Ansonsten wäre, um ein Beispiel von Forgó,
Krügel und Reiners~(2008, S.~19) aufzunehmen, jede Sachinformation über
die Stadt Hannover personenbezogen, da wir sehr einfach Personen
bestimmen können, die in Hannover wohnen.

Für die Verarbeitung personenbezogener Daten gilt seit dem 25. Mai 2018
die DS-GVO als unmittelbar geltendes Recht. Sie wird ergänzt durch das
jeweils anwendbare Datenschutzgesetz (Landesdatenschutzgesetze für
öffentliche Stelle in Trägerschaft der Länder, Bundesdatenschutzgesetz
für Private und öffentliche Stellen in Trägerschaft des Bundes) sowie
fachgesetzliche Regelungen, zum Beispiel der Statistik-, Sozial-,
Archiv-, Geodaten- und Umweltinformationsgesetze. Dabei haben im Sinne
des Subsidiaritätsprinzips spezialgesetzliche Regelungen Vorrang vor den
allgemeinen Datenschutzgesetzen. Grundsätzlich gilt, dass
personenbezogene Daten ohne Einwilligung der betroffenen Person nur zu
bestimmten Zwecken verarbeitet werden dürfen (Art.~6, Abs.~1 DS-GVO).
Die Verarbeitung bestimmter \enquote{besonderer Kategorien}
personenbezogener Daten ist grundsätzlich verboten und nur in
Ausnahmefällen erlaubt (Art.~9, Abs.~1-2 DS-GVO). Bei der Verarbeitung
personenbezogener Daten sind die Grundsätze des Art.~5 DS-GVO zu
beachten, die betroffenen Personen haben weitgehende Rechte (Kapitel~III
DS-GVO), während den verarbeitenden Stellen diverse Pflichten auferlegt
werden (Kapitel~IV DS-GVO). Eine kurze Übersicht zu den Grundlagen des
Datenschutzrechts nach In-Kraft-Treten der DS-GVO im Allgemeinen bietet
zum Beispiel Hoeren~(2018, S.~405ff).

Für den Umgang mit personenbezogenen Daten in der Forschung sind sowohl
allgemeine als auch forschungsspezifische Regelungen der jeweils
anwendbaren Rechtsgrundlagen relevant. Dabei geht es zum einen um die
Frage, ob die Verarbeitung rechtmäßig ist, zum anderen um Grundsätze der
Verarbeitung (zum Beispiel frühestmögliche Anonymisierung
beziehungsweise Pseudonymisierung von Daten, die nicht anonymisiert
werden können). Eine kompakte Übersicht über Anforderungen an den
Datenschutz in der Forschung in Deutschland bietet RatSWD~(2017).
Allerdings werden dort die Auswirkungen der DS-GVO noch nicht
berücksichtigt. Für ausführliche Informationen zum Datenschutz in der
Forschung wird noch immer regelmäßig auf Metschke und Wellbrock~(2002
{[}1994{]}) verwiesen. Die Auswirkungen der DS-GVO auf die Forschung
thematisiert Schaar~(2016). Bevor Forschende jedoch konkrete gesetzliche
Regelungen in Anspruch nehmen, welche die Verarbeitung personenbezogener
Daten erlauben, müssen sie zunächst selbst prüfen, ob die von ihnen
angestrebte Verarbeitung für Forschungszwecke einer Abwägung der
betroffenen Grundrechte auf informationelle Selbstbestimmung und
Freiheit der Forschung standhält. Die Anforderungen dieser Abwägung
stellen RatSWD~(2017, S.~6f) sowie Metschke und Wellbrock~(2002,
S.~10--12) dar.

Für Gruppen natürlicher Personen oder juristische Personen gilt das
Datenschutzrecht nicht. Falls sich jedoch aus der Erhebung, Verarbeitung
oder dem Bekanntwerden von Daten ein Nachteil~(zum Beispiel
Stigmatisierung oder ein wirtschaftlicher Schaden) für eine Gruppe
natürlicher Personen oder für juristische Personen ergeben kann, ist aus
Gründen der Forschungsethik eine entsprechende Abwägung sowie die
sichere Verarbeitung der Daten ebenfalls geboten. Gegebenenfalls ist
eine Ethikkommission zu befassen.

Die Literatur zum Datenschutz in der Forschung und zur Forschungsethik
hat oft die sozialwissenschaftliche, zeithistorische,
humanwissenschaftliche oder medizinische Forschung im Blick.
Dementsprechend liegt der Fokus oft auf den strengen Anforderungen für
die Verarbeitung der \enquote{besonderen Kategorien} von
personenbezogenen Daten. Hinweise zu einem datenschutzrechtlich und
forschungsethisch adäquaten Umgang mit in den Erd- und
Umweltwissenschaften typischen Arten von Daten, die zum Beispiel über
den Zustand von Flächen Rückschlüsse auf Vermögensverhältnisse von
Eigentümerinnen oder die Bewirtschaftungspraxis von Landnutzerinnen
zulassen, enthält diese Literatur nicht. Auch über die
spezialgesetzlichen Regelungen im Bereich Zugang zu Geodaten und
Umweltinformationen gibt die Literatur nur selten Auskunft, im Gegensatz
zu den Regelungen in den Bereichen Statistik, Sozialstaat, Gesundheit
und Archivwesen.

\hypertarget{personenbezug-von-raumbezogenen-daten-insbesondere-geodaten---grundsuxe4tzliches}{%
\subsection{Personenbezug von raumbezogenen Daten, insbesondere Geodaten
-
Grundsätzliches}\label{personenbezug-von-raumbezogenen-daten-insbesondere-geodaten---grundsuxe4tzliches}}

Für die Erd- und Umweltwissenschaften sind raumbezogene Daten von
besonderer Bedeutung. Wird in der Literatur die Verarbeitung von Daten
mit gleichzeitigem Personen- und Raumbezug in der Forschung behandelt,
geht es in der Regel um die Verwendung anfallender Nutzungs- und
Standortdaten mobiler Geräte und Anwendungen insbesondere in den Sozial-
und Technikwissenschaften (etwa bei Hilty~\emph{et al.}~2012) oder um
die Explikation und Nutzung des Raumbezugs klassischer wirtschafts- und
sozialstatistischer Daten (wie in RatSWD~2012). Solche Daten sind
unbestreitbar personenbezogen und können gegebenenfalls auch besondere
Kategorien personenbezogener Daten enthalten. Dagegen sind für die Erd-
und Umweltwissenschaften eher raumbezogene Daten zur natürlichen und
bebauten Umwelt~(im folgenden: Geodaten) relevant. Diese sind seit
längerem Gegenstand von Auseinandersetzungen um die Reichweite des
Datenschutzrechts.

Die rechtsdogmatische Diskussion wurde zunächst im Vorfeld der
Einführung von Informationsrechten im Umweltbereich geführt (zum
Beispiel Taeger~1991; Raum~1993) und setzte sich dann im Zuge des durch
die INSPIRE-Richtlinie mandatierten Aufbaus von öffentlich zugänglichen
Datenplattformen für behördliche Geodaten fort (Karg und Weichert~2007,
S.~5-11). Auch in der Praxis legen Gerichte und Aufsichtsbehörden teils
sehr unterschiedliche Maßstäbe an (Karg und Weichert~2007, S.~11--19).
Anlass zu weiteren Diskussionen gab das Aufkommen von Internetdiensten
wie Google Street View (vergleiche Gusy und Eichendorfer~2018, Rn.~47).
So haben sich Geodaten zu einem wichtigen Beispiel entwickelt, an Hand
dessen Abgrenzungsprobleme zwischen personenbezogenen und
nicht-personenbezogenen Daten grundsätzlich verhandelt werden (Ende der
2000er Jahre zum Beispiel durch Weichert~2009; Forgó und Krügel~2010;
aktuell zum Beispiel bei~Krügel 2017; Schild~2018). Dies macht eine
Rezeption der Rechtslage für juristische Laien nicht einfach und kann
den Blick auf teilweise bestehende schulenübergreifende
Übereinstimmungen in der praktischen Beurteilung verstellen.

Weitgehende Einigkeit scheint jedoch über die folgenden Grundsätze zu
herrschen: Bei Geodaten handelt es sich in erster Linie um Sachdaten.
Diese \enquote{sind zunächst keine personenbezogenen Daten. Sie beziehen
sich auf eine Sache und beschreiben diese} (Schild~2018, Rn.~22). Gerade
\enquote{Geodaten können aber auch Personenbezug aufweisen}, was bei
Katastern und katasterartigen Informationssystemen, über die sich zu
Grundstücksflächen gehörige Adressen und Eigentümerinnen ermitteln
lassen, offensichtlich wird (Schild~2018, Rn.~23). In weniger
offensichtlichen Fällen kommt es neben der Genauigkeit des Raumbezugs
auch auf den Informationsgehalt der Daten an, da dieser bestimmt, ob die
Sachdaten Auswirkungen \enquote{auf rechtliche, wirtschaftliche oder
soziale Positionen einer Person haben oder sich zur Beschreibung ihrer
individuellen Verhältnisse} haben (Schild~2018, Rn.~24; vergleiche auch
Artikel-29-Datenschutzgruppe~2007, S.~13). Ein typischer Fall dafür aus
der kommerziellen Verwendung von Geodaten ist das Geoscoring, also die
Bewertung der Kreditwürdigkeit einer Person über auf ihre Wohnadresse
bezogene Sachdaten (Schild~2018, Rn.~25--27).

Damit widerspricht der aktuelle Konsens der früher verbreiteten
Einschätzung, dass grundsätzlich alle lagegenauen Geodaten
personenbezogen seien, weil durch sie eine sachbezogene Information
einer identifizierbaren Person, nämlich der Eigentümerin oder Nutzerin
der Fläche, zugeordnet werden kann (so etwa Taeger~1991). Eine solche
Überdehnung des Schutzbereichs des Datenschutzes ist weder praktikabel,
noch unterstützt sie einen konsequenten Vollzug, sondern leistet im
Gegenteil Argumenten für eine Aushöhlung eines vollkommen überdehnten
Datenschutzes Vorschub (Weichert~2007, S.~23). Dass dies nicht nur
theoretische Überlegungen sind, zeigen beispielsweise
Beratungsergebnisse der Kommission Recht und Geodaten der Deutschen
Gesellschaft für Kartographie (DGfK), die nicht nur für einen deutlich
engeren Schutzbereich bei Fernerkundungsdaten plädieren, sondern selbst
für Katasterdaten jeglichen Personenbezug bestreiten~(Diez~\emph{et
al.}~2009).

\hypertarget{personenbezug-von-raumbezogenen-daten-insbesondere-geodaten-bisher-vorgeschlagene-pruxfcfverfahren}{%
\subsection{Personenbezug von raumbezogenen Daten, insbesondere Geodaten
-- bisher vorgeschlagene
Prüfverfahren}\label{personenbezug-von-raumbezogenen-daten-insbesondere-geodaten-bisher-vorgeschlagene-pruxfcfverfahren}}

Konkrete Prüfverfahren für den Personenbezug von raumbezogenen Daten
wurden im Kontext der Umsetzung der INSPIRE-Richtlinie zunächst von zwei
einschlägigen Gutachten etabliert (Karg~2008; Forgó, Krügel und
Reiners~2008), deren Ergebnisse in begleitenden Zeitschriftenartikeln
weiter kontextualisiert wurden (Weichert~2009; Forgó und Krügel~2010).
In beiden Verfahren von steht zu Anfang die Prüfung, ob die Daten sich
auf eine identifizierbare Person beziehen lassen (Karg~2008, S.~8--16;
Forgó, Krügel und Reiners~2008, S.~10--17, 22--23). Falls ja, ist zu
prüfen, ob identifizierbare Geodaten tatsächlich Informationen über eine
Person sind. Dazu wird im Anschluss an die
Artikel-29-Datenschutzgruppe~(2007) ein so genannter kontextbezogener
Ansatz verfolgt (Karg~2008, S.~16--22; Forgó, Krügel und Reiners~2008,
S.~18--21). Das heißt, es werden Arten und Weisen unterschieden, wie
primär sachbezogene Geodaten Auskunft über Personen geben und sich
\enquote{auf die Rechte und Interessen einer natürlichen Person
auswirken} können (Artikel-29-Datenschutzgruppe~2007, S.~13). Dabei muss
ein Datum nur in eine der genannten Kategorien beziehungsweise Kontexte
fallen, um personenbezogen zu sein und damit dem Datenschutz zu
unterfallen.

Es werden drei so genannte Kontexte oder Elemente identifiziert
(Karg~2008, S.~21--22; Forgó, Krügel und Reiners~2008, S.~19):

\begin{itemize}
\item
  \textbf{Inhaltskontext beziehungsweise Inhaltselement}\\
  Es handelt sich um Daten, die direkt Auskunft über Einzelpersonen
  geben, zum Beispiel über Identität, Individualität, Intimsphäre, aber
  auch über Zustände oder tatsächliches Verhalten der Betroffenen und
  damit offensichtlich im klassischen Sinne um personenbezogene Daten.
  Bei diesen Daten kann es sich gegebenenfalls sogar um besondere
  Kategorien personenbezogener Daten handeln. Typisches Beispiel sind
  Ortungsdaten aus dem Mobilfunk.
\item
  \textbf{Zweckkontext beziehungsweise Zweckelement}\\
  Die Daten sind geeignet, das räumliche Umfeld einer Person zu bewerten
  und können so dazu führen, dass diese Person in einer bestimmten Weise
  beurteilt, behandelt oder beeinflusst wird. Dazu müssen die Daten
  nicht flächenscharf sein, wohl aber kleinräumig. Ein typisches
  Beispiel ist das bereits genannte Geoscoring.
\item
  \textbf{Ergebniskontext beziehungsweise Ergebniselement}\\
  Das Bekanntwerden dieser Daten kann sich auf Rechte und Interessen
  einer bestimmten Person auswirken. Dabei geht es meist um
  wirtschaftliche Interessen und Verfügungsrechte über Eigentum. Die
  Person muss dabei spezifisch, d.~h. im Unterschied zu anderen
  betroffen sein: Die Information, dass Spitzbergen abgelegen ist und
  ein rauhes Klima hat, ist sicherlich relevant für dortige
  Grundstückspreise, jedoch offensichtlich nicht personenbezogen. Ein
  typisches Beispiel für den Ergebniskontext ist bei weiter Auslegung
  bereits die Information, ob einem Gebäude Denkmalwert zukommt, und
  zwar auf Grund der damit verbundenen Beschränkung der
  Verfügungsbefugnis über Eigentum (Karg~2008, S.~21); bei enger
  Auslegung dagegen in etwa erst die Information, ob die Verursachung
  einer Altlast auf einem Grundstück dem aktuellen Eigentümer
  zuzurechnen ist, nicht aber die reine Information über das
  Vorhandensein einer Umweltbelastung an einem bestimmten Ort
  (Weichert~2007, S.~22).
\end{itemize}

Auch wenn Geodaten in all diesen Fällen dem Datenschutz unterliegen, ist
der entstehende Eingriff in Grundrechte unterschiedlich schwerwiegend.
Er ist bei vorliegenden Inhaltselementen größer als bei Zweckelementen
und bei Ergebniselementen am kleinsten (Karg~2008, S.~21--22). Diese
Risikoabschätzung in zweifacher Hinsicht relevant: Zum einen betrifft
sie die grundrechtliche Abwägung, ob die Erhebung und Verarbeitung der
Daten nicht übermäßig und verhältnismäßig ist. Zum anderen hat sie
Einfluss darauf, was angemessene Schutzmaßnahmen sind und welche
gesetzlichen Ausnahmen zur Erhebung, Verarbeitung und Weitergabe
gegebenenfalls auch ohne Einverständnis der Betroffenen in Betracht
kommen. Eine solche abgestufte Beurteilung wird auch durch
Erwägungsgrund 26 DS-GVO gestützt, der einen \enquote{graduellen} Ansatz
bei der Feststellung der Identifizierbarkeit unterstützt (vergleicheGusy
und Eichendorfer~2018, Rn.~47; Hofmann und Johannes~2017, S.~222--225).

Für die Bereitstellung von amtlichen Geodaten für die Wirtschaft nimmt
das Gutachten von Karg (2008, S.~55--69) zudem für zahlreiche Geodaten
eine allgemeine Einschätzung des Risikos nach einem Ampelsystem vor.
Dabei fasst Karg~(2008) die Reichweite des Ergebniskontexts deutlich
weiter als beispielsweise Weichert~(2007, S.~21--22) -- und das, obwohl
es sich um ein kooperierendes Autorenteam handelt (vergleiche Karg und
Weichert 2007). Gerade in diesem für die Erd- und Umweltwissenschaften
besonders relevanten Bereich besteht also wenig Einigkeit.

\hypertarget{personenbezug-von-geodaten---spezialgesetzliche-regelungen-und-verwaltungsleitfuxe4den}{%
\subsection{Personenbezug von Geodaten - Spezialgesetzliche Regelungen
und
Verwaltungsleitfäden}\label{personenbezug-von-geodaten---spezialgesetzliche-regelungen-und-verwaltungsleitfuxe4den}}

Für den Zugang zu bei Behörden vorliegenden personenbezogenen Geo- und
Umweltdaten bestehen spezialgesetzliche Regelungen. Das
Umweltinformationsgesetz (UIG) und die analogen Regelungen der Länder
begründen ein Jedermannsrecht auf Zugang zu bei Behörden vorliegenden
Umweltinformationen. Dabei wird der Datenschutz gegenüber den
allgemeineren Regelungen des Informationsfreiheitsgesetzes (IFG) und den
analogen Gesetzen der Länder deutlich eingeschränkt (Karg und
Weichert~2007, S.~23--24). Obwohl bei der massenhaften öffentlichen
Bereitstellung von Geodaten für automatisierte Abrufverfahren ein ganz
anderes Gefährdungspotenzial vorliegt als beim Einblick in einzelne
Akten auf Antrag (BFDI~2009, S.~89--90) und Einzelfallentscheidungen
kaum möglich sind (Karg 2008, S.~42), verweist die Umsetzung der
INSPIRE-Richtlinie in Deutschland, das Geodaten-Zugangsgesetz (GeoZG),
auf die Regelungen des UIG.

Als Grundlage für die notwendigen Entscheidungen in den Verwaltungen
wurde durch eine Bund-Länder-Arbeitsgruppe ein mit dem Bundesbeauftragen
für Datenschutz und Informationsfreiheit abgestimmter Leitfaden
entwickelt (IMAGI~2014). Dieser stellt vorsichtshalber unter
Vernachlässigung des Kriteriums des inhaltlichen Bezugs ausschließlich
auf die Identifizierbarkeit ab und geht daher davon aus, dass mit
Ausnahme reiner Geländemodelle (ohne Abbildung von Gebäuden,
Grundstücksgrenzen, etc.) jedes hinreichend lagegenaue Geodatum
Personenbezug habe. Da aber auch schwierig zu beurteilen ist, ab welcher
Genauigkeit die Bestimmbarkeit gegeben ist, sei vorsichtshalber bei
allen Geodaten ein Personenbezug zu vermuten (IMAGI~2014, S.~8, A5).
Zugleich wird aber empfohlen davon auszugehen, dass bei Daten mit
gröberem Raumbezug das öffentliche Interesse an den Daten die
schützenswerten Interessen der Betroffenen im Regelfall überwiegt (und
zwar unabhängig davon, ob die Maßstäbe des IFG oder des UIG/GeoZG
anzulegen sind), so dass einer Weitergabe beziehungsweise öffentlichen
Bereitstellung der Daten nichts entgegen stehe (IMAGI~2014, S.~11--13,
A5). Die Verwaltungspraxis vermeidet also mit Hilfe einer argumentativen
Krücke die Bewertung im Einzelfall und kann daher für den Umgang mit
Geodaten, die im Rahmen von Forschungsprojekten entstehen keine
Orientierung bieten.

\hypertarget{diskussion-und-zwischenfazit}{%
\subsection{Diskussion und
Zwischenfazit}\label{diskussion-und-zwischenfazit}}

Die praxisbezogene Literatur zum Datenschutz in der Forschung behandelt
schwerpunktmäßig Themen, die in den Erd- und Umweltwissenschaften kaum
eine Rolle spielen. Dem Umgang mit \enquote{regulären} (nicht den
\enquote{besonderen Kategorien} zuzurechnenden) personenbezogenen Daten
wird vergleichsweise wenig Raum gewidmet. Zudem sind die für die Erd-
und Umweltwissenschaften besonders wichtigen Geodaten in der
juristischen Fachliteratur zum Beispiel geworden, an dem
rechtsdogmatische Differenzen ausgetragen werden. Dies erschwert
Laiinnen die Rezeption der Rechtslage. Die praxisorientierte Literatur
sowie die Rechts- und Verwaltungspraxis beim Thema Datenschutz und
Geodaten hat sich bisher vor allem mit systematisch und oft
duldungspflichtig erhobenen Verwaltungsdaten, ihrer massenhaften
Bereitstellung für die Öffentlichkeit und ihrer Nachnutzung in der
Wirtschaft einerseits sowie mit von kommerziellen Anbietern öffentlich
bereitgestellten Satellitenbildern und Lichtbildern von Wohnhausfassaden
andererseits befasst. Aus dieser Konstellation abgeleitete implizite
Annahmen prägen die Argumentationsweise ebenso wie Vermutungen über den
datenschutzrechtlichen Status einzelner Arten von Datensätzen. Am
deutlichsten wird dies in den pauschalen und sehr weitgehenden
Vermutungen über den Personenbezug bestimmter Arten von behördlichen
Geodaten, die der IMAGI-Leitfaden anstellt (IMAGI~2014). Aber auch in
den Gutachten von Karg~(2008) sowie Forgó, Krügel und Reiners~(2008)
werden einige für die Forschung relevante Arten von Geodaten gar nicht,
andere übermäßig pauschal behandelt. Letztes gilt insbesondere für so
genannte Punktdaten, denen unabhängig von ihrem Inhalt grundsätzlich
Identifizierbarkeit und eine hohe Aussagekraft zugeschrieben wird.

Für den Umgang mit im Rahmen von Forschung erhobenen Geodaten und ihre
Bereitstellung zur Nachnutzung bietet die Literatur daher kaum
Anknüpfungspunkte. Nichtsdestotrotz liegen Vorschläge für Prüfverfahren
zum Personenbezug von Geodaten von Karg~(2008) sowie Forgó, Krügel und
Reiners~(2008) vor, die trotz Verankerung in verschiedenen
rechtsdogmatischen Schulen zu praktisch sehr ähnlichen Prüfabläufen
kommen. An diese könnten zu entwickelnde praxisorientierte
Handreichungen zum Umgang mit Geodaten aus der Forschung anschließen,
mit Hilfe derer offensichtlich unproblematische Fälle, offensichtlich
problematische Fälle und Zweifelsfälle vergleichsweise einfach
unterschieden werden könnten. Dies könnte zur Verbesserung der Praxis im
Umgang mit potenziell personenbezogenen Geodaten in der Forschung
beitragen, ohne dass für allzu viele Fälle der Aufwand einer
vollständigen juristischen Prüfung nötig wird.

\hypertarget{vertrauliche-daten-in-der-forschungspraxis-der-erd--und-umweltwissenschaften}{%
\section{Vertrauliche Daten in der Forschungspraxis der Erd- und
Umweltwissenschaften}\label{vertrauliche-daten-in-der-forschungspraxis-der-erd--und-umweltwissenschaften}}

Ziel des empirischen Teils dieser Untersuchung war es, relevante
Fallgruppen von potenziell personenbezogenen Daten in den Erd- und
Umweltwissenschaften und von offenen Fragen im Umgang mit diesen zu
ermitteln. Dazu wurden leitfadengestützte, auf die Erhebung von
Prozesswissen zielende Gespräche mit Expertinnen geführt (fundierende,
informatorische Experteninterviews im Sinne Bogners~2014, S.~22-25).
Deren relevante Inhalte wurden im Sinne einer qualitativen
Inhaltsanalyse zusammengefasst. Wie für Vorstudien dieser Art von
Bogner~(2014, S.~39-41) empfohlen, wurde direkt aus der Tonaufzeichnung
exzerpiert. Dabei wurde jedes Vorkommen von als problematisch
empfundenen Daten als ein Fall behandelt. Zur Gruppierung der Fälle
wurde das Verfahren der qualitativen Typenbildung (Keller und
Kluge~2010) angewendet.

Die Auswahl der Gesprächspartnerinnen erfolgte kriterienorientiert nach
maximaler Variation in den drei Merkmalen \enquote{Art der Organisation}
(Hochschule oder außeruniversitäre Forschungseinrichtung),
\enquote{Rolle der befragten Person} (Forschung oder
Forschungsunterstützung) und \enquote{Wissenschaftsgebiet} (Landschafts-
und standortbezogene Forschung in den DFG-Fachge\-bieten
34~Geowissenschaften und 23~Agrar- und Forstwissenschaften; andere Arten
von Forschung im DFG-Fachgebiet 34;~oder ökologische und
naturschutzbezogene Forschung im DFG-Fachgebiet 21~Biologie).
Potenzielle Interviewpartnerinnen wurden über eine systematische Suche
auf den Websites relevanter universitärer Institute und
außeruniversitärer Forschungseinrichtungen in Nord- und Ostdeutschland
gefunden. Die Rekrutierung von Arbeitsgruppenleiterinnen aus der
Forschung als Gesprächspartnerinnen gestaltete sich schwieriger, während
Kolleginnen mit direktem Bezug zu Infrastruktur-Einheiten meist
gesprächsbereit waren und zwar auch dann, wenn sie selbst (auch)
Forschung betreiben.

Im Rahmen dieser Vorstudie konnten Gespräche mit insgesamt sechs
Expertinnen geführt werden, die teils aus mehreren Positionen berichten
konnten, so dass die Bereiche universitäre Arbeitsgruppe und
außeruniversitäre Arbeitsgruppe je zwei Mal, der Bereich
außeruniversitäre Infrastruktur vier Mal abgedeckt war. Dabei war die
landschaftsbezogene Forschung drei Mal, die Geowissenschaften im engeren
Sinne zwei Mal sowie Ökologie und Naturschutz ein Mal vertreten.

Die Interviews zeigen, dass von Erd- und Umweltwissenschaftlerinnen als
(zumindest potenziell) vertraulich bewertete Daten sehr divers sind. In
den Interviews lassen sich insgesamt 20 Fallgruppen vertraulicher Daten
in den Erd- und Umweltwissenschaften identifizieren. Von diesen haben
elf Bezug zur Fragestellung dieser Untersuchung, da personenbezogene
Daten betroffen sein können. Die überwiegende Zahl dieser in die
Typologie aufgenommenen Fallgruppen kommen aus der landschaftsbezogenen
Forschung, unabhängig von ihrer genauen disziplinären Verortung. Aus dem
Bereich der Geowissenschaften trägt vor allem die sich ebenfalls auf die
menschliche Nutzung gerichtete Forschung zu Georisiken und Naturgefahren
zur Konstituierung der Typen bei. Ökologie und Naturschutz sind
ebenfalls nur dann betroffen, wenn es um den Zusammenhang mit
Landnutzung geht.

Aus diesen elf relevanten Fallgruppen ließen sich analytisch sechs
(Interview-übergreifende) Typen ableiten. Dabei teilen jeweils zwei
Typen viele gemeinsame Eigenschaften, so dass sich eine Darstellung in
drei Typgruppen (hier identifiziert durch Großbuchstaben A, B, C) mit je
zwei Typen anbietet. Da die Typologie empirisch basiert ist (die Daten
also so einteilt, wie eine idealisierte Interviewpartnerin es hätte tun
können), sind die Kriterien zur Ausgrenzung der Typen nicht über die
gesamte Typologie einheitlich. So wird als Kriterium teils die Art der
Identifizierbarkeit, teils der Informationsgehalt der Daten und teils
die methodische Provenienz der Daten herangezogen. Im Folgenden werden
zunächst Ergebnisse dargestellt, die über die einzelnen konstruierten
Typen hinweg für alle der Typologie zugeordneten Fallgruppen Gültigkeit
besitzen. Anschließend werden die einzelnen Typen vorgestellt.

In den verbleibenden neun Fallgruppen geht es um andere Arten von
Vertraulichkeit von Daten als (potenziellen) Personenbezug
beziehungsweise berechtigte Interesse Betroffener. Die aus ihnen
gebildeten Typen werden im Anschluss an die Vorstellung der Typologie
nur mit kurzer Beschreibung gelistet, eine ausführliche Analyse liegt
außerhalb des Untersuchungsbereichs dieses Beitrags.

\hypertarget{typen-uxfcbergreifende-ergebnisse}{%
\subsection{Typen-übergreifende
Ergebnisse}\label{typen-uxfcbergreifende-ergebnisse}}

Über alle sechs konstruierten Typen hinweg lässt sich feststellen, dass
einige Befragte große Unsicherheit darüber äußern, welche Daten
schutzwürdig sind, während andere sich in ihren Einschätzungen
weitgehend sicher fühlen. Verschiedene Befragte, die sich in ihren
Einschätzungen relativ sicher sind, kommen aber zu durchaus
unterschiedlichen Beurteilungen des Status vergleichbarer Daten.

Die Wahrnehmung von Daten als vertraulich wird von den Befragten
allerdings meist nicht mit einer Argumentation begründet, die sich im
engeren Sinne auf das Persönlichkeitsrecht bezieht. Dies zeigt sich auch
daran, dass Betriebe und teils auch Personengruppen nicht
notwendigerweise als weniger schutzwürdig angesehen werden als
Einzelpersonen. Auch die Frage der Identifizierbarkeit spielt für die
Befragten nicht die Hauptrolle. Vielmehr halten die Befragten Daten vor
allem dann für vertraulich, wenn eine nicht-intendierte
außerwissenschaftliche Nutzung das Potenzial hat, betroffenen Dritten
(insbesondere Eigentümerinnen beziehungsweise Landnutzerinnen) zu
schaden. Dafür werden zwei mögliche Arten von Szenarien angegeben:

\begin{itemize}
\item
  Die Daten könnten zeigen, dass die Nutzungs- und
  Bewirtschaftungspraxen von Landnutzerinnen nicht der guten Praxis
  folgen oder sogar unzulässig sind.
\item
  Die Daten könnten, auch ohne dass ein Bezug zum Verhalten der
  aktuellen Nutzerinnen besteht, negative Auswirkungen auf
  wirtschaftliche Interessen der Betroffenen haben (zum Beispiel auf
  Grund von Naturgefahren oder bestehenden Umweltbelastungen).
\end{itemize}

Die Wahrscheinlichkeit einer Rezeption dieser Daten außerhalb der
Wissenschaft wird jedoch als in den meisten Fällen sehr gering
eingeschätzt. In die unterschiedlichen Einschätzungen der
Vertraulichkeit von Daten scheint auch einzufließen, inwieweit die
Forschenden etwaige Landnutzerinnen als Forschungssubjekte sehen und
sich ihnen daher forschungsethisch verpflichtet fühlen oder ob diese gar
Kooperationspartner sind, denen sich die Forschenden nicht nur
verpflichtet fühlen, sondern von denen sie auch abhängig sind.

Daten, bei denen ein solches vornehmlich wirtschaftliches
Schadenspotenzial nicht gesehen wird, scheinen oft als grundsätzlich
harmlos angesehen zu werden. Die -- aus Perspektive der
Sozialwissenschaften selbstverständlich erscheinenden --
persönlichkeits- und datenschutzrechtlichen Herausforderungen bei der
Verarbeitung von Daten, die aus anderen Gründen personenbezogen sind,
werden daher teils gar nicht erkannt. Gute Beispiele sind hier die
Veröffentlichung von Transkripten qualitativer Interviews oder der
Umgang mit Daten, die Auskunft über das Wohnumfeld bestimmbarer Personen
geben.

In den Antworten auf die Frage, wie mit potenziell problematischen Daten
dieser Art umgegangen wird, fallen folgende Aspekte auf:

\begin{itemize}
\item
  Die Befragten äußern sich vor allem zur Veröffentlichung von Daten. Ob
  die Daten überhaupt hätten erhoben und verarbeitet werden dürfen,
  spielt dagegen seltener eine Rolle -- wohl auch, weil das Thema erst
  im Umfeld von Datenveröffentlichungen virulent geworden ist.
\item
  Es bestehen Unsicherheiten und Defizite im Bereich der Einwilligung.
  Eine solche wird in der Regel eingeholt, allerdings erfolgt diese oft
  mündlich, die Qualität schriftlicher Einverständniserklärungen ist
  uneinheitlich und eine Nachnutzung meist nicht mit abgedeckt.
\item
  Auch wo Daten als vertraulich eingeschätzt werden, wird von
  technischen und organisatorischen Maßnahmen der Datensicherheit (wie
  Zugangsbeschränkung, Pseudonymisierung beziehungsweise formale
  Anonymisierung oder Verschlüsselung) bei der Datenverarbeitung in der
  Arbeitsgruppe während des Forschungsprozesses nicht berichtet.
\item
  Die Befragten an Universitäten berichten, keine Möglichkeit zur
  institutionellen Archivierung vertraulicher Daten zu haben
  beziehungsweise zu kennen. An den außeruniversitären
  Forschungseinrichtungen gibt es zusätzlich zum öffentlichen
  Repositorium teils interne Datenbanken, in denen auch vertrauliche
  Forschungsdaten abgelegt werden. Die in solchen Fällen übliche
  Zugangsbeschränkung ist, dass die ursprüngliche Projektverantwortliche
  beziehungsweise AG Leiterin darüber entscheidet, wem auf Anfrage
  Zugang gewährt wird. Regelungen für den Fall, dass diese nicht mehr
  verfügbar ist, scheinen nicht üblich zu sein. Gelegentlich werden
  gerade seit längerem vorhandene (\enquote{legacy}) Datensammlungen in
  solchen internen Repositorien als der Einrichtung gehörend verstanden
  und die Zugangsgewährung erfolgt durch die Leitung des Bereichs.
\item
  In den Aussagen zur Datenveröffentlichung werden meist nur die
  Optionen Open-Access-Veröffentlich\-ung versus komplettes Zurückhalten
  der Daten thematisiert. Als mögliche Zugangsbeschränkung für
  veröffentlichte Daten werden nur Embargo-Perioden und Lizenzverträge
  für kommerzielle Daten genannt, wissenschaftsfreundliche Modelle für
  den beschränkten Zugang zu Daten (wie sie etwa an den
  RatSWD-zertifizierten Datenzentren üblich sind) sind weitgehend
  unbekannt.
\item
  Befragte berichten von der Erfahrung, dass im Bereich Open Science
  engagierte Editorinnen und Gutachterinnen sich in Einzelfällen mit
  Verweis auf die sich entwickelnde disziplinäre \emph{best practice}
  teils auch dann schwer tun, Fachartikel ohne
  Open-Access-Veröffentlich\-ung der zugehörigen Daten zu akzeptieren,
  wenn diese Daten aus Sicht der Autorinnen begründbar vertraulich sind.
  Gleichzeitig sind aber sowohl Journals als auch Förderorganisationen
  in der Regel auch mit der Publikation von Daten zufrieden, die so
  stark aufbereitet sind, dass sie gar nicht mehr sinnvoll nachnutzbar
  sind.
\end{itemize}

\hypertarget{typgruppe-a-abschuxe4tzbare-risiken-bei-geo--und-umweltdaten-aus-naturwissenschaftlicher-feldforschung}{%
\subsection{Typgruppe A: Abschätzbare Risiken bei Geo- und Umweltdaten
aus naturwissenschaftlicher
Feldforschung}\label{typgruppe-a-abschuxe4tzbare-risiken-bei-geo--und-umweltdaten-aus-naturwissenschaftlicher-feldforschung}}

Ein Großteil der potenziell zum Nachteil der Betroffenen
identifizierbaren Daten, von denen die Befragten berichten, stammt aus
naturwissenschaftlicher Feldforschung. Es geht also um Daten, die
entweder direkt im Gelände erhoben wurden oder aus Analyse im Feld
genommener Proben stammen. Dabei kann es sich um sehr verschiedene Daten
handeln, fast das gesamte Methodenspektrum der Erd- und
Umweltwissenschaften ist betroffen: Es geht sowohl um experimentelle
Daten (aus Feldversuchen) als auch um Beobachtungsdaten. Es kann sich um
qualitative Daten (Zuordnungen zu Klassen wie Bodentypen oder
pflanzensoziologische Einheiten), ebenso handeln wie um quantitative
Daten (von Zählungen von Wildtiersichtungen über Messungen
physikalischer und chemischer Parameter in einzelnen
Umweltkompartimenten bis hin zu Erkundungen von Oberfläche und
Untergrund mit geophysikalischen Methoden). Zeitlich kann es sich um
Einzelerhebungen (Ergebnisse von Einzelaufnahmen beziehungsweise von
einmaligen Mess- beziehungsweise Beobachtungskampagnen) oder um
Zeitreihen (zeitlich diskrete longitudinale Erhebungen oder
quasi-kontinuierliche Daten aus automatisierten Messstationen, so
genannten Datenloggern) handeln.

Die Befragten bestätigen, dass es für viele dieser Daten plausible
Nachnutzungsszenarien gibt. Der zunehmende Druck seitens der
Fördermittelgeber, möglichst alle in Projekten beziehungsweise über
geförderte Infrastrukturen erhobene Daten zu veröffentlichen, sowie die
Forderung renommierter Zeitschriften, zumindest die direkt zur
Textpublikation gehörenden Daten zu veröffentlichen, wird daher
insgesamt positiv beurteilt. Im Zuge der weiteren Verbreitung, des
Ablegens von Daten in Repositorien und des Aufbaus neuer fachbezogener
Dateninfrastrukturen, wird aber durchaus auch wahrgenommen, dass sich
diese Daten auf verschiedene Art und Weise in Bezug zur Eigentümerin
beziehungsweise Nutzerin der betroffenen Fläche setzen lassen und dass
dies in bestimmten Fällen eine gewisse Vertraulichkeit der Daten
begründen sollte. Da es sich um Daten aus Feldforschung handelt, die
eine begrenzte Zahl von Parametern aufnehmen, erscheint eine Abschätzung
des Risikos in der Regel dabei in erster Näherung möglich.

Potenziell identifizierbare und gegebenenfalls vertrauliche Geo- und
Umweltdaten aus naturwissenschaftlicher Feldforschung unterscheiden sich
wesentlich in der Art und Weise der möglichen (Re-)Identifikation und
den jeweiligen Möglichkeiten der Anonymisierung. \textbf{Typ A1} fasst
durch genauen Raumbezug (Koordinaten oder bestimmte Fläche)
identifizierende Daten zusammen. Es handelt es sich im Wesentlichen um
Daten, die in Geoinformatik und Kartographie als thematische Daten, im
geodatenbezogenen Behördendeutsch als Fachdaten bezeichnet werden.
\textbf{Typ~A2} beschreibt durch Inhalt und Kontext des Datensatzes
identifizerbare Daten. Dies ist insbesondere dann der Fall, wenn die
untersuchten Standorte Teil einer kleinen Grundgesamtheit sind. Eine
Identifizierbarkeit kann in diesem Fall nicht nur aus veröffentlichten
Daten erwachsen, sondern sich bereits aus der (Text-)Publikation von
Ergebnissen ergeben. Wo es nötig und möglich ist, eine
(Re-)Identifikation durch Anonymisierungsmaßnahmen zu verhindern,
schränken diese für beide Typen die Nachnutzbarkeit der Daten in der
Regel massiv ein.

\hypertarget{typgruppe-b-daten-mit-schwer-abzuschuxe4tzenden-risiken}{%
\subsection{Typgruppe B: Daten mit schwer abzuschätzenden
Risiken}\label{typgruppe-b-daten-mit-schwer-abzuschuxe4tzenden-risiken}}

Bestimmte Daten wurden nur von einem Teil der Befragten als
problematisch identifiziert, bereiten aus deren Sicht jedoch besondere
Probleme. Dabei geht es um Daten, bei denen nur schwer einzuschätzen
ist, welche Informationen sich aus ihnen ableiten lassen und das Risiko
sich daher kaum bestimmen lässt.

In \textbf{Typ~B1} geht es um zusätzliche Risiken, die sich durch
möglich werdende Verknüpfungen von vielen, für sich genommen zunächst
unproblematisch erscheinenden Datensätzen in einem zukünftigen Open
Data-Regime in den Erd- und Umweltwissenschaften ergeben können. In
\textbf{Typ~B2} geht es dagegen um Daten, die auf Grund der
Erhebungsmethode einen extrem hohen Informationsgehalt haben, so dass
vorab kaum geklärt werden kann, welche Angaben sich aus den Daten
ableiten lassen, aber vermutet werden muss, dass sich evtl.
problematische Rückschlüsse ziehen lassen. Dies betrifft insbesondere
selbst erhobene kleinräumige Fernerkundungsdaten, gegebenenfalls aber
auch hochauflösende beobachtende Zeitreihen von terrestrisch erhobenen
Messwerten. Gleichzeitig besteht in beiden Fällen auf Grund des hohen
Informationsgehalts besonderes Interesse an einer Verfügbarkeit der
Daten für die Nachnutzung.

\hypertarget{typgruppe-c-daten-aus-nicht-naturwissenschaftlichen-methoden}{%
\subsection{Typgruppe C: Daten aus nicht-naturwissenschaftlichen
Methoden}\label{typgruppe-c-daten-aus-nicht-naturwissenschaftlichen-methoden}}

In allen Bereichen der Erd- und Umweltwissenschaften kommen nach
Auskunft der Befragten immer wieder auch nicht-naturwissenschaftliche
Methoden zum Einsatz. Bei den so generierten Daten handelt es sich
teilweise um \enquote{klassische} personenbezogene Daten, wie sie aus
den Wirtschafts- und Sozial- und Kommunikationswissenschaften bekannt
sind. In den Erd- und Umweltwissenschaften werden auch innerhalb
hauptsächlich naturwissenschaftlich ausgerichteter Arbeitsgruppen
Befragungen durchgeführt, Dokumente analysiert, konventionelle und
soziale Medien ausgewertet sowie statistische Wirtschafts- und
Sozialdaten nachgenutzt oder auch betriebs-, flächen- oder
gebäudebezogene quantitative Daten selbst erhoben.

Dabei umfasst \textbf{Typ C1} Wirtschafts- und Sozialdaten als
Paradaten, die zur Begleitung und Ergänzung naturwissenschaftlich
ausgerichteter Studien erhoben werden. Entscheidend ist, dass diese
Daten in Verbindung mit den oben diskutierten (raumbezo genen)
naturwissenschaftlichen Daten stehen, die ihrerseits schlecht zu
anonymisieren sind und daher identifizierend wirken können. Diese zur
Kontextualisierung naturwissenschaftlicher Untersuchungen erhobenen
Daten haben oft kein eigenständiges Nachnutzungspotenzial. Dagegen steht
\textbf{Typ C2} für eigenständige wirtschafts-, sozial- und
kommunikationswissenschaftliche Daten. Auch die Fragestellungen sind
dann, trotz des naturwissenschaftlichen Kontexts, im weiteren Sinn
wirtschafts-, sozial- und kommunikationswissenschaftlich geprägt. Oft
wird für solche Untersuchungen entweder mit Gruppen aus den betreffenden
Disziplinen kooperiert oder Personen mit entsprechender Expertise
eingestellt. Hier hängt das Nachnutzungspotenzial von der Art der
erhobenen Daten ab.

\hypertarget{weitere-problemfelder}{%
\subsection{Weitere Problemfelder}\label{weitere-problemfelder}}

Neben den in die Typologie aufgenommenen Fallgruppen, bei denen Rechte
und Interessen Dritter betroffen sind, wurden auch andere
Konstellationen genannt, in denen Daten als vertraulich eingeschätzt
werden. Da diese Punkte die Fragestellung dieser Arbeit nur am Rande
berühren, werden Sie im folgenden nur kurz genannt und in der weiteren
Diskussion nicht behandelt.

Unerwünschte außerwissenschaftliche Nachnutzung raumbezogener Geo- und
Umweltdaten kann in Einzelfällen nicht nur Personen betreffen, sondern
auch \emph{Gefahr für Umwelt und Natur} bergen. Dies gilt insbesondere
für Daten aus der Naturschutzökologie, zum Beispiel zu Vorkommen
geschützter Arten. Manche Daten, insbesondere aus den Geowissenschaften,
haben \emph{unmittelbar außerwissenschaftliche, insbesondere}
\emph{wirtschaftliche Bedeutung}. Dies kann die erhebenden Einrichtungen
vor Zielkonflikte (Verwertung vs.~offene Daten) stellen oder ein Risiko
für die Reputation der Disziplin beziehungsweise der Einrichtung
darstellen (zum Beispiel in Konflikten um Ressourcen oder Landnutzung).
Werden Daten veröffentlicht, geht dies zwangsläufig mit dem
\emph{Verlust der Kontrolle} über ihre Interpretation einher. Während
dies im wissenschaftlichen Diskurs zumindest für Messdaten explizit
gewünscht ist, wird dies bei außerwissenschaftlicher Nachnutzung gerade
für komplexe Datenprodukte teils als Problem gesehen. Insbesondere für
räumlich explizite Modelle und bei der Erstellung räumlich expliziter
Datenprodukte wie digitalen Karten werden oft Daten aus diversen Quellen
zusammengeführt und verarbeitet, die sehr unterschiedlichen
Lizenzbedingungen unterliegen. Dies kann die \emph{Lizenzierung von
Datenprodukten zur Nachnutzung} erschweren. Bei \emph{Daten aus
Industriekooperation} sperren sich die privaten Partnerinnen oft
dagegen, sich vorab zu verpflichten, alle im Rahmen eines Projekts
erhobenen Daten zu veröffentlichen und zwar auch dann, wenn diese
offensichtlich keine Betriebsgeheimnisse enthalten.

\hypertarget{diskussion-und-zwischenfazit-1}{%
\subsection{Diskussion und
Zwischenfazit}\label{diskussion-und-zwischenfazit-1}}

Die Ergebnisse des empirischen Teils bestätigen, dass auch in den -- im
Diskurs um personenbezogene Forschungsdaten weithin als unverdächtig
geltenden -- Erd- und Umweltwissenschaften tatsächlich Daten anfallen,
die Personen zugeordnet werden können, zum Beispiel Eigentümerinnen und
Nutzerinnen von Flächen. Insofern solche identifizierbaren Daten auch im
datenschutzrechtlichen Sinne Personenbezug haben, wird es sich in der
Regel um gewöhnliche personenbezogene Daten handeln. In Einzelfällen
können aber durchaus \enquote{besondere Kategorien} personenbezogener
Daten nach Art.~9 DS-GVO anfallen und/oder weitere Rechtsgebiete
betroffen sein, sei es bei Ortungsdaten von Landmaschinen in Forschung
zur Präzisionslandwirtschaft (Arbeitsrecht) oder bei Fotos aus Citizen
Science-Projekten (Urheberrecht).

Besonders betroffen sind die Teildisziplinen, die landschafts- und
landnutzungsbezogene Forschung betreiben. In anderen Teilbereichen der
Erd- und Umweltwissenschaften sind Daten eher aus anderen Gründen
vertraulich. Diese anderen Fälle sind für die fachspezifische
Unterstützung im Bereich Forschungsdatenmanagement durchaus relevant und
sollten in eventuellen fachspezifischen Handreichungen berücksichtigt
werden, können hier jedoch nicht weiter diskutiert werden.

Die Typen-übergreifenden Ergebnisse zeigen, dass die Forschenden den
Personenbezug ihrer Daten selten in Isolation in Betracht ziehen.
Vielmehr trennen sie den Personenbezug in ihren impliziten
forschungsethischen Abwägungen und in ihren durch das Interview
veranlassten Explikationen nicht von anderen berechtigten Interessen
Dritter. Die Interessen von Personenmehrheiten und juristischen Personen
von den Befragten nicht als grundsätzlich weniger schützenswert
angesehen als die Interessen von Einzelpersonen. Zudem machen die
Befragten den Umfang des als schützenswert angesehenen Bereichs auch vom
Verhältnis zwischen Forschenden und Betroffenen abhängig. Beides ist
forschungsethisch einleuchtend, geht aber über eine rein
datenschutzrechtliche Betrachtung hinaus. Zugleich werden jedoch andere
Dimensionen von Persönlichkeitsrelevanz, etwa dass auch erd- und
umweltwissenschaftliche Daten in Einzelfällen zum Beispiel Auskunft über
das private Wohnumfeld geben könnten, von den Befragten oft nicht
gesehen. Die Überlegungen der Befragten zur Vertraulichkeit von Daten
erfolgen vor allem im Kontext von Datenpublikationen, die Frage nach den
Anforderungen an ihre Erhebung und Bedingungen ihrer Verarbeitung stellt
sich ihnen seltener. Wo Daten tatsächlich personenbezogen sind, wird die
Forschungspraxis den strengen Anforderungen an Information der
Betroffenen, Einverständniserklärungen und Grundsätze der Verarbeitung
wohl in der Regel nicht gerecht.

Es besteht also einerseits über den engen Geltungsbereich des
Datenschutzrechtes hinaus Klärungsbedarf im Bereich von Forschungs-,
Publikations- und Datenethik in den Erd- und Umweltwissenschaften. Die
Frage, wie hier eine gute Praxis im Umgang mit vertraulichen Daten
entwickelt werden kann, lässt sich nicht auf das Datenschutzrecht und
technisch-organisatorische Maßnahmen der Datensicherheit reduzieren.
Andererseits besteht im Hinblick auf die Anforderungen des Datenschutzes
sowohl ein Sensiblitäts- und Informationsdefizit auf Seiten der
Forschenden als auch ein Defizit an Beratungskompetenz und
Infrastrukturen bei den disziplinären Datenzentren.

Für verschiedene Typen von Daten sind weitere typspezifische Ergebnisse
relevant. Zunächst ist festzuhalten, dass teils auch im Rahmen
vornehmlich naturwissenschaftlich ausgerichteter Studien «ganz
klassische» personenbezogene Daten beziehungsweise Daten, welche auch
ohne Ansehen ihres Raumbezugs die berechtigten Interessen Dritter
betreffen, erhoben werden (Typgruppe~C). Während betriebswirtschaftliche
Daten klar als vertraulich eingeschätzt werden, scheint der Umgang mit
den mittels sozialwissenschaftlicher Methoden gewonnenen Daten
uneinheitlich und insbesondere bei den nicht einfach zu anonymisierenden
qualitativen Daten nicht immer angemessen zu sein.

Ein weiteres Cluster von Antworten bezieht sich auf Daten, deren Risiko
besonders schlecht einzuschätzen ist (Typgruppe~B). Hier äußern
Forschende Unverständnis, dass sie alleine für die Unbedenklichkeit der
von ihnen bei Repositorien abgelieferten Daten verantwortlich sein
sollen. Gleichzeitig ist auch bei den Infrastrukturen beschäftigten
Personen unklar, wie sie bei einer solchen Beurteilung unterstützen
sollten. Es wird einerseits zugestanden, dass diese Daten nicht im Open
Access veröffentlicht werden sollten und andererseits beklagt, dass
anscheinend nur in Ausnahmefällen eine nachnutzbare Archivierung und
Zugänglichmachung erfolgt.

Der größte Komplex von genannten Fällen bezieht sich jedoch auf die bei
einzelnen Datensätzen gegebenenfalls in Kombination mit einfach zu
ermittelnden Adress- und Katasterdaten vorliegenden Risiken bei Geo- und
Umweltdaten aus naturwissenschaftlicher Feldforschung (Typgruppe~A).
Hier unterscheiden die Befragten klar zwei Typen: Daten, bei denen
formale Anonymisierung -- also die Entfernung direkter Identifikatoren
einschließlich exaktem Raumbezug -- ausreicht, um faktische Anonymität
sicherzustellen (Typ~A1) und Daten, die auch bei formaler Anonymisierung
identifizierbar bleiben (Typ~A2). Dabei wird jedoch betont, dass viele
Daten in anonymisierter Form kaum noch nutzbar sind. Dies gilt auch für
rein formale Anonymisierung, da Daten in der Regel nur im Kontext
interpretierbar sind. Darüber, welche dieser Daten überhaupt
schützenswert sind, besteht jedoch keine Einigkeit zwischen den
Befragten und bei manchen Befragten auch deutliche individuelle
Unsicherheit. Im Wesentlichen ist strittig, in welchem Umfang ein Datum
Rückschlüsse auf die wirtschaftliche Lage oder Details des Verhaltens
einer Nutzerin oder der Eigentümerin zulassen muss beziehungsweise wie
viel Potenzial, den Wert eines Grundstücks zu beeinflussen, das Datum
haben muss, um personenbezogen zu sein oder -- falls Personenmehrheiten
oder juristische Personen betroffen sind -- aus forschungsethischen
Gründen als vertraulich zu gelten. Damit besteht in der
forschungsethischen Bewertung durch die Akteure Unklarheit an demselben
Punkt, der auch in der juristischen Fachliteratur strittig ist.

Einige Befragte beschreiben die begrenzte Nachnutzbarkeit weit
aufbereiteter Daten, wie sie oft in Supplementen enthalten sind. Dies
gilt verstärkt für Daten, die anonymisiert wurden. Archivierung bei
Infrastrukturen mit Recherchierbarkeit der Metadaten und
\emph{scientific use} (begrenzter Zugriff auf den vollständigen
Datensatz zu geprüften Forschungszwecken nur nach Abschließen eines
Lizenzvertrags) gibt es im Bereich der Erd- und Umweltwissenschaften
bisher aber fast ausschließlich für (in Bezug auf Rohstoffe)
wirtschaftlich relevante Daten aus den Geowissenschaften der festen
Erde.

Insbesondere an Hochschulen scheint es oft keine Möglichkeit zum
institutionellen Archivieren vertraulicher Daten zu geben, da in der
Regel die Unbedenklichkeit der Daten bei Übergabe an einen
Archivierungsdienst bescheinigt werden muss. Werden Daten zur Sicherung
der guten wissenschaftlichen Praxis oder zur Nachnutzung innerhalb einer
Arbeitsgruppe aufgehoben, aktuell aber nicht benötigt, ist jedoch gerade
für vertrauliche Daten eine sichere institutionelle Datenhaltung statt
Datenhaltung im Workspace der Arbeitsgruppe angezeigt.

\hypertarget{handlungsbedarf}{%
\section{Handlungsbedarf}\label{handlungsbedarf}}

Aus den Ergebnissen der Literaturauswertung zum Datenschutz in der
Forschung, insbesondere bei Geodaten, und den Ergebnissen der Vorstudie
zum Umgang mit potenziell personenbezogenen Daten in den Erd- und
Umweltwissenschaften ergibt sich ein Bedarf, die Datenpraxis im Rahmen
der Disziplinkultur zu verbessern. Vorrangig ist dies natürlich eine
Aufgabe der Fächer, zum Beispiel im Rahmen wissenschaftsgeleiteter
Weiterentwicklungsprozesse wie der Nationalen
Forschungsdateninfrastruktur~(NFDI), in der Datenschutz und Datenethik
als \enquote{cross-cutting issue} behandelt werden sollen.\footnote{Vergleiche
  die unter
  \url{https://www.dfg.de/en/research_funding/programmes/nfdi/} zur
  Verfügung gestellten Unterlagen und Glöckner \emph{et al.~}(2019).}
Die Ergebnisse zeigen jedoch auch mehrere Handlungsfelder für
Infrastruktureinrichtungen auf und legen bestimmte Handlungsoptionen
nahe.

In folgenden drei Themenbereichen ist \textbf{Beratungsbedarf der
Forschenden} sichtbar geworden, den Infrastruktureinrichtungen
adressieren sollten:

\begin{itemize}
\item
  Einschätzung des Personenbezugs beziehungsweise der Vertraulichkeit
  auf Grund berechtigter Interessen Dritter bei im Rahmen von
  Forschungsvorhaben erhobenen Geodaten.
\item
  Umgang mit wirtschafts- und sozialwissenschaftlichen Paradaten zu
  naturwissenschaftlichen Untersuchungen.
\item
  Einhaltung der Anforderungen an die Erhebung und Verarbeitung
  personenbezogener Daten in Forschungsprojekten~(nicht erst an
  Weitergabe und Veröffentlichung).
\end{itemize}

Der Umfang der relevanten Informationen legt nahe, hier eine
eigenständige Handreichung zu erarbeiten. Der geeignete Moment im
Datenlebenszyklus, um diese Themen frühzeitig zu berücksichtigen, ist
die Projektplanung. Daher sollte diese Handreichung mit vorliegenden und
entstehenden disziplinspezifischen Informationen zu
Datenmanagementplänen abgestimmt werden. Zur Einschätzung des
Personenbezugs bei Geodaten aus der Forschung bietet es sich an, eine
Anleitung zur Risikoabschätzung in Anlehnung an die existierenden, mit
Blick auf Behördendaten entwickelten Prüfverfahren zu entwickeln.

Im Bereich der Herstellung von \textbf{Archivierbarkeit
personenbezogener und anderweitig vertraulicher Daten in
institutionellen Datenrepositorien} käme jede entsprechende Verbesserung
nicht nur den Erd- und Umweltwissenschaften zu Gute, sondern allen
Disziplinen, die nicht alle ihre vertraulichen Daten bei fachbezogenen
Infrastrukturen abgeben können. Hier sind zunächst die institutionellen
Datenarchive beziehungsweise Repositorien insbesondere der Hochschulen
angesprochen. Sollte sich herausstellen, dass gute Praxis im Umgang mit
vertraulichen Daten in institutionellen Datenarchiven und Repositorien
(zumindest an Hochschulen) tatsächlich um ein bundesweites Desiderat und
kein Problem einzelner Einrichtungen handelt, würde sich ein
entsprechende Forschungs- und Entwicklungsprojekt an einer gut
aufgestellten Einrichtung anbieten. Wünschenswert wäre es dabei, die
besonderen Bedingungen bei kooperativen und gehosteten Diensten wie
RADAR mit zu betrachten.\footnote{Zu RADAR
  vgl.~\url{https://www.radar-service.eu/de}.}

Wenn es in Zukunft möglich sein soll, auch \textbf{personenbezogene und
anderweitig vertrauliche Daten aus den Erd- und Umweltwissenschaften zu
publizieren}, statt nur institutionell zu archivieren, das heißt die
Daten auffindbar und für \emph{scientific use} begrenzt zugänglich zu
machen, wäre es notwendig, bei disziplinären Datenzentren entsprechende
Modi des Zugangs als Standard-Dienstleistung mit definierten
Zugangsbedingungen zu etablieren. Dies setzt jedoch nicht nur die
Entwicklung von Workflows, sondern auch den Aufbau von Expertise und die
Entwicklung von disziplinspezifischen Richtlinien zu vertraulichen Daten
und Datenschutz, effektiver Anonymisierung und vertraglichen Bedingungen
der Zugangsgewährung voraus.

Die \textbf{Risiken der Kombination von Datensätzen} aus den Erd-,
Umwelt- und Agrarwissenschaften sind zurzeit auch für Datenexpertinnen
bezüglich der Infrastrukturen kaum überschaubar. Es besteht
Klärungsbedarf dazu, welche inhaltlichen Rückschlüsse über den
Informationsgehalt der Einzeldatensätze hinausgehenden und aus der
Kombination vieler identifizierbarer, aber als Einzeldatensatz harmloser
Forschungsdaten mit den offenen Verwaltungsdaten im Geo-, Umwelt- und
Agrarförderungsbereich sowie anderen leicht verfügbaren Daten
(Adressverzeichnisse, Kataster) tatsächlich getroffen werden können. Es
liegt nahe, dieses Thema im Zusammenhang mit Projekten zur Schaffung von
Strukturen für \emph{scientific use} von Forschungsdaten der Erd- und
Umweltwissenschaften zu bearbeiten.

\hypertarget{fazit-und-ausblick}{%
\section{Fazit und Ausblick}\label{fazit-und-ausblick}}

Am Beispiel der Erd- und Umweltwissenschaften habe ich aufgezeigt, dass
auch in scheinbar \enquote{unverdächtigen} Disziplinen personenbezogene
Forschungsdaten vorkommen. Die Ergebnisse der empirischen Vorstudie
zeigen eine ganze Reihe verschiedener Arten personenbezogener
Forschungsdaten auf, die in der Forschungspraxis der Erd- und
Umweltwissenschaften eine Rolle spielen. Sie legen außerdem nahe, dass
der Umgang mit ihnen in der Forschungspraxis auf Grund der mangelnden
Vertrautheit mit dem Datenschutz nicht immer den rechtlichen
Anforderungen entspricht. Auch Unterstützung in Form
disziplinspezifischer Handreichungen, qualifizierter Beratung oder
institutionalisierte Möglichkeiten, Daten sicher zu archivieren und
gegebenenfalls zugangsbeschränkt zu publizieren, bestehen kaum. Zudem
ergibt sich aus der Auswertung der Literatur, dass bisherige Antworten
auf die Frage, wann Geodaten unter den Datenschutz fallen, die
spezifischen Erfordernisse erd- und umweltwissenschaftlicher
Forschungsdaten nicht berücksichtigen. Allgemeine Handreichungen zum
Datenschutz in der Forschung geben keine Auskunft, Handreichungen zum
Datenschutz bei Geodaten der Verwaltung sind nicht auf die Forschung
anwendbar und selbst unter Fachjuristinnen herrscht Uneinigkeit über die
Bewertung. Aus dieser Situation ergeben sich Herausforderungen an die
Weiterentwicklung der disziplinären Datenkultur und
Dateninfrastrukturen. Dabei können die Infrastruktureinrichtungen
wertvolle Unterstützung leisten, wie die dargestellten Handlungsoptionen
zeigen. Insbesondere die Kooperation zwischen Forschenden und
Infrastruktureinrichtungen im Rahmen des NFDI-Prozesses bietet hier gute
Möglichkeiten, diese Herausforderungen anzugehen.

\hypertarget{anmerkungen}{%
\section{Anmerkungen}\label{anmerkungen}}

Diese Veröffentlichung basiert auf einer Masterarbeit im weiterbildenden
Masterstudiengang Bibliotheks- und Informationswissenschaft~(Library and
Information Science, M. A.~(LIS)) im Fernstudium am Institut für
Bibliotheks- und Informationswissenschaft der Humboldt-Uni\-ver\-si\-tät zu
Berlin aus dem Sommersemester 2018. Ich danke Prof.~Dr.~Wolfram
Horstmann (Staats- und Universitätsbibliothek Göttingen und
Humboldt-Universität zu Berlin) und Maxi Kindling (damals
Humboldt-Universität zu Berlin) herzlich für die Betreuung der Arbeit.
Angewandte Forschung dieser Art ist unmöglich ohne die \enquote{kindness
of strangers} der Interviewpartnerinnen, die sich auf ein Gespräch über
potenziell problematische Aspekte ihrer Forschungspraxis beziehungsweise
ihrer forschungsunterstützenden Arbeitspraxis in der Infrastruktur
einließen. Auch ihnen sei an dieser Stelle herzlich gedankt.

Der Autor dieses Beitrags ist kein Jurist, dementsprechend enthält der
Text \emph{keine} verbindlichen Auskünfte zu Rechtsthemen. Diese Arbeit
nutzt das generische Femininum, wo sich genderneutrale
Partizipialkonstruktionen nicht anbieten. Alle Links wurden am
23.09.2019 zuletzt geprüft, soweit nicht anders vermerkt.

Angaben zur Datenverfügbarkeit: Die Aufzeichnungen der Gespräche wurden
nach Abschluss der Untersuchung gelöscht. Die der Auswertung zu Grunde
liegenden Exzerpte sind für Dritte nicht nachnutzbar und lassen sich
nicht verlässlich anonymisieren. Sie werden daher nicht zugänglich
gemacht.

\hypertarget{bibliographie}{%
\section{Bibliographie}\label{bibliographie}}

Artikel-29-Datenschutzgruppe~(2007). \emph{Stellungnahme 4/2007 zum
Begriff \enquote{personenbezogene Daten}.} WP 136. Brüssel:
Datenschutzgruppe nach Artikel 29 der Richtlinie 95/46/EG.

Bernard, Lars \emph{et al.}~(2016). \enquote{Geodateninfrastrukturen}.
In:~\emph{Handbuch der Geodäsie}. Hrsg. von Willi Freeden und Reiner
Rummel. Springer Reference Naturwissenschaften. Berlin, Heidelberg:
Springer. \url{https://doi.org/10.1007/978-3-662-46900-2_66-1}.

Bertelmann, Roland~(2017). \enquote{Geowissenschaften}.
In:~\emph{Praxishandbuch Open Access.} Hrsg. von Konstanze Söller und
Bernhard Mittermeier. Berlin, Boston: De~Gruyter. Kap.~5c, S.~261--265.

Bertelmann, Roland \emph{et al.}~(2014). \emph{Einstieg ins
Forschungsdatenmanagement in den Geowissenschaften.} Handreichung.
Potsdam, Berlin: DFG-Projekt EWIG.
\url{https://doi.org/10.2312/lis.14.01}.

BFDI~(2009). \emph{Tätigkeitsbericht zum Datenschutz für die Jahre 2007
und 2008.} 22.~Tätigkeitsbericht. Berlin: Bundesbeauftragter für den
Datenschutz und die Informationsfreiheit.

Blühdorn, Ingolfur~(2007) Sustaining the unsustainable: Symbolic
politics and the politics of simulation. \emph{Environmental Politics}
16. 2, S.~251-275, \url{https://doi.org/10.1080/09644010701211759}

Bogner, Alexander~(2014). \emph{Experteninterviews: Theorien, Methoden,
Anwendungsfelder.} Wiesbaden: VS~Verlag für Sozialwissenschaften.

COPDESS~(2015). \emph{COPDESS Statement of Commitment: Statement of
Commitment from Earth and Space Science Publishers and Data Facilities.}
Coalition on Publishing Data in the Earth und Space Sciences.
\url{http://www.copdess.org/statement-of-commitment/} .

Diepenbroek, Michael und Gerold Wefer~(2011).
\enquote{Forschungsdateninfrastrukturen in den Bio- und
Geowissenschaften}. In: \emph{Zeitschrift für Bibliothekswesen und
Bibliographie} 58.3--4, S.~167--171.
\url{https://doi.org/10.3196/18642950115834125}.

Diez, Dietrich \emph{et al.}~(2009). \enquote{Schutz des
Persönlichkeitsrechts bei der Verarbeitung von Geodaten}. In: \emph{zfv}
134.6, S.~357--362.

Forgó, Nikolaus und Tina Krügel~(2010). \enquote{Der Personenbezug von
Geodaten: Cui bono, wenn alles bestimmbar ist?} In: \emph{MultiMedia und
Recht} 12.1, S.~17--23.\\
\url{https://beck-online.beck.de/Bcid/Y-300-Z-MMR-B-2010-S-17-N-1}.

Forgó, Nikolaus, Tina Krügel und Nico Reiners~(2008). \emph{Forschungs-
und Entwicklungsauftrag zum Thema Geoinformation und Datenschutz}.
Rechtsgutachten. Hannover: Leibniz Universität Hannover, Institut für
Rechtsinformatik im Auftrag der X GmbH.
\url{https://iri.uni-hannover.de/tl_files/pdf/Geodaten_Datenschutz_Gutachten.pdf}.

Glöckner, Frank Oliver \emph{et al.}~(2019). \emph{Berlin Declaration on
NFDI Cross-Cutting Topics.} Zenodo.
\url{http://doi.org/10.5281/zenodo.3457213}

Gusy und Eichendorfer~(2018). \enquote{BDSG 2018 §1}. In: \emph{BeckOK
Datenschutzrecht.} Hrsg. von Wolff und Brink. 23. Aufl. Stand:
25.07.2017. Beck-Online, Rn.~27--50.

Hanson, Brooks, Kerstin Lehnert und Joel Cutcher-Gershenfeld~(2015).
\enquote{Committing to Publishing Data in the Earth and Space Sciences}.
In: \emph{EOS Earth \& Space Science News} 96.
\url{https://doi.org/10.1029/2015EO022207}.

Hilty, Lorenz \emph{et al.}~(2012). \emph{Lokalisiert und identifiziert:
Wie Ortungstechnologien unser Leben verändeRn.~}TA-SWISS.~Zürich: vdf
Hochschulverlag der ETH. \url{https://doi.org/10.3218/3477-6}.

Hoeren, Thomas~(2018). \emph{Internetrecht.} Vorlesungsskript. Münster:
Universität Münster, Institut für Informations-, Telekommunikations- und
Medienrecht. \url{https://www.itm.nrw/lehre/materialien/}.

Hofmann, Johanna M und Paul C Johannes~(2017). \enquote{DS-GVO:
Anleitung zur autonomen Auslegung des Personenbezugs}. In:
\emph{Zeitschrift für Datenschutz} 7.5, S.~221--226.
\url{https://beck-online.beck.de/Bcid/Y-300-Z-ZD-B-2017-S-221-N-1}.

IMAGI~(2014). \emph{Behördenleitfaden zum Datenschutz bei Geodaten und
-diensten.} Leitfaden. Interministerieller Ausschuss für
Geoinformationswesen.
\url{https://www.imagi.de/IMAGI/DE/Arbeitsgruppen/AG-Geodatenschutz/ag-geodatenschutz_node.html}.

Karg, Moritz~(2008). \emph{Datenschutzrechtliche Rahmenbedingungen für
die Bereitstellung von Geodaten für die Wirtschaft.} Rechtsgutachten.
Kiel: Unabhängiges Landeszentrum für Datenschutz Schleswig-Holstein im
Auftrag der GIW-Kommission. url:
\url{https://www.bmwi.de/Redaktion/DE/Publikationen/Geobusiness/datenschutzstudie-2-die-ampelstudie.html}.

Karg, Moritz und Thilo Weichert~(2007). \emph{Datenschutz und
Geoinformation.} Rechtsgutachten. Kiel: Unabhängiges Landeszentrum für
Datenschutz Schleswig-Holstein im Auftrag des Bundesministeriums für
Wirtschaft und Technologie.
\url{https://www.bmwi.de/Redaktion/DE/Publikationen/Geobusiness/datenschutzstudie-1-datenschutz-und-geoinformation.html}.

Kelle, Udo und Susann Kluge~(2010 {[}1999{]}). \emph{Vom Einzelfall zum
Typus: Fallvergleich und Fallkontrastierung in der qualitativen
Sozialforschung.} 2., überarb. Aufl. Bd.~15. Qualitative
Sozialforschung. Wiesbaden: VS Verlag für Sozialwissenschaften.

Klump, Jens~(2012). \enquote{Geowissenschaften}. In:
\emph{Langzeitarchivierung von Forschungsdaten: Eine Bestandsaufnahme}.
Hrsg. von Heike Neuroth \emph{et al.} Boizenburg: Verlag Werner
Hülsbusch. \url{https://nestor.sub.uni-goettingen.de/bestandsaufnahme/}.

Krügel, Tina~(2017). \enquote{Das personenbezogene Datum nach der
DS-GVO: Mehr Klarheit und Rechtssicherheit?} In: \emph{Zeitschrift für
Datenschutz} 7.10, S.~455--460.
\url{https://beck-online.beck.de/Bcid/Y-300-Z-ZD-B-2017-S-455-N-1}.

McNutt, Marcia \emph{et al.}~(2016). \enquote{Liberating field science
samples and data}. In: \emph{Science} 351.6277, S.~1024--1026.
\url{https://doi.org/10.1126/science.aad7048}.

Metschke, Rainer und Rita Wellbrock~(2002 {[}1994{]}). \emph{Datenschutz
in Wissenschaft und Forschung.} 3.~überarb. Aufl. Berlin: Berliner
Beauftragter für Datenschutz und Informationsfreiheit, Hessischer
Datenschutzbeauftragter.

RatSWD~(2012). \emph{Georeferenzierung von Daten: Situation und Zukunft
der Geodatenlandschaft in Deutschland.} Abschlussbericht der AG
\enquote{Georeferenzierung von Daten}. Berlin: Scivero Verlag für den
Rat für Wirtschafts- und Sozialdaten.
\url{https://www.ratswd.de/publikationen/georeferenzierung-von-daten}.

--~(2017). \emph{Handreichung Datenschutz.} Output Series 5. Berlin: Rat
für Wirtschafts- und Sozialdaten.
\url{https://doi.org/10.17620/02671.6}.

Raum, Bertram~(1993). \enquote{Umweltschutz und Schutz personenbezogener
Daten}. In: \emph{Computer und Recht} 9, S.~162--170.

Schaar, Katrin~(2016). \emph{Was hat die Wissenschaft beim Datenschutz
künftig zu beachten?} RatSWD Working Paper 257. Aktualisierte Fassung,
Erstpublikation in Zeitschrift für Datenschutz, 6.9~(2016), 224ff.
Berlin: Rat für Wirtschafts- und Sozialdaten.
\url{https://doi.org/10.17620/02671.19}.

Schild~(2018). \enquote{DS-GVO Artikel 4: Begriffsbestimmungen}. In:
\emph{BeckOK Datenschutzrecht.} Hrsg. von Wolff und Brink. 23. Aufl.
Stand: 01.02.2018. Beck-Online, Rn.~22--27.

Taeger, Jürgen~(1991). \enquote{Umweltschutz und Datenschutz}. In:
\emph{Computer und Recht} 7, S.~681--688.

Vines \emph{et al.} (2013). The Availability of Research Data Declines
Rapidly with Article Age. \emph{Current Biology} 24, S.~94-97.
\url{https://doi.org/10.1016/j.cub.2013.11.014}

Weichert, Thilo~(2007). \enquote{Der Personenbezug von Geodaten}. In:
\emph{Datenschutz und Datensicherheit} 31.1, S.~17--23.
\url{https://doi.org/10.1007/s11623-007-0006-3}.

--~(2009). \enquote{Geodaten: Datenschutzrechliche Erfahrungen,
Erwartungen und Empfehlungen}. In: \emph{Datenschutz und
Datensicherheit} 33.6, 347ff.
\url{https://doi.org/10.1007/s11623-009-0071-x}.

%autor
\begin{center}\rule{0.5\linewidth}{0.5pt}\end{center}

\textbf{Niklas Hartmann} studierte Umweltwissenschaften in Potsdam und
Wageningen, Niederlande sowie Wissenschaftssoziologie in Lancaster,
Großbritannien. Seit 2016 ist er Koordinator Forschungsdaten an der
Universität Potsdam und Fachreferent in der Universitätsbibliothek.

\url{https://orcid.org/0000-0002-7328-3625}

\end{document}
